\documentclass[12pt,a5paper,landscape]{report}

\usepackage[left=1cm,right=1cm,top=1.5cm,bottom=1cm,bindingoffset=0cm,pdftex]{geometry}
\usepackage[utf8]{inputenc}
\usepackage{indentfirst}

\usepackage{color}
\usepackage{listings}
\usepackage{amsmath}
\usepackage{textcomp} % Для значка градуса \textdegree
\usepackage{amssymb}  % Для значка треугольника
\usepackage{amsmath}

\usepackage[russian]{babel}

\begin{document}
\parindent=1cm
\pagestyle{empty}

\clearpage
\section*{Рубежный контроль №1}
\subsection*{Вариант I}
\subsubsection*{Задача №1}
Задан куб с длиной ребра $a$. Вычислить площадь одной грани куба $S_f$, площадь полной поверхности куба $S$ и объём куба $V$.
\subsubsection*{Задача №2}
Найти наименьшее среди трёх целых чисел $p$, $q$ и $r$.
\subsubsection*{Задача №3}
Дано натуральное число $n$ и $n$ действительных чисел $a_1, a_2, ... , a_n$. Найти количество чётных чисел $a_1, a_2, ... , a_n$.

\clearpage
\section*{Рубежный контроль №1}
\subsection*{Вариант II}
\subsubsection*{Задача №1}
Задан конус с длиной образующей $l$ и радиусом основания $R$. Вычислить объём конуса $V = \pi R l$ и площадь полной поверхности $S = \pi R (R + l)$.
\subsubsection*{Задача №2}
Для заданного $x$ вычислить значение функции $f(x)$.
\begin{equation*}
f(x) = 
\begin{cases}
\frac {1}{x^2}&\text{, при $x < -1$;}\\
x^2&\text{, при $-1 <= x <= 2$;}\\
4&\text{, при $x > 2$.}\\
\end{cases}
\end{equation*}
\subsubsection*{Задача №3}
Задано действительное число $a$ и натуральное число $n$. Вычислить: $3a^n$; $\frac{1}{a+1} + \frac{1}{a+2} + ... + \frac{1}{a+n-1} + \frac{1}{a+n}$. (Не использовать модуль $Math$ и функцию \texttt{Power}.)

\end{document}