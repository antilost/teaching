\documentclass[10pt,a4paper]{report}
\usepackage[left=0.5cm,right=0.5cm,top=0.5cm,bottom=1cm,bindingoffset=0cm,pdftex]{geometry}
\usepackage[utf8]{inputenc}
\usepackage[russian]{babel}
\usepackage{indentfirst}
\usepackage{color}
\usepackage{listings}
\usepackage{amsmath}
\usepackage{textcomp} % Для значка градуса \textdegree
\usepackage{amssymb}  % Для значка треугольника
\usepackage{amsmath}
\begin{document}
\parindent=1cm
\pagestyle{empty}
\lstset{ language=Pascal, basicstyle=\small\ttfamily, numbers=left, numberstyle=\tiny, stepnumber=1, numbersep=5pt, extendedchars=\true, showstringspaces=false, breakatwhitespace=true, frame=single, keepspaces=true }

\clearpage
\section*{Обработка файлов}
\textbf{Файл (англ. file)} --- именованная область данных на носителе информации. Носителем информации может быть любое устройство, позволяющее выполнять чтение/запись и долговременное хранение информации, например, самый распространённый сейчас магнитный жёсткий HDD-диск (англ. hard magnetic disk drive) или более дорогой, но быстрее выполняющий чтение/запись твердотельный SSD-накопитель (англ. solid-state drive). Носитель информации, диск, имеет область данных как последовательность бит, кодирующих содержимое файла; и к этой последовательности бит можно обратиться, зная имя файла: прочитать или перезаписать файл или создать новый файл, указав файловой системе (и впоследствии диску) имя нового файла и его содержимое.

\subsection*{Классификация файлов}
В зависимости от содержимого файла разделяют текстовые и двоичные (бинарные) файлы.
\subsubsection*{1. Текстовые файлы}
\textbf{Текстовый файл} --- это файл, содержимое которого --- текст, --- последовательность символов, обычно группируемых в строки. Структура текстовых файло обычно интуитивно понятна человеку, и такие файлы можно легко создавать и редактировать с помощью обычного текстового редактора (например "Блокнот" в Windows).

\paragraph*{Пример текстового файла onegin.txt.} Это текстовый файл, в котором записан текст вступления к известному произведению А. С . Пушкина. Структура \texttt{onegin.txt} файла фактически определяется рифменной схемой: отдельные сонеты по 14 строк, разделенные пустыми строками, примечаниями и эпиграфами; но по отдельности рассматривать каждую строку нельзя.
\begin{verbatim}
ГЛАВА ПЯТАЯ

О, не знай сих страшных снов
     Ты, моя Светлана!
Жуковский.

I
В тот год осенняя погода
Стояла долго на дворе,
Зимы ждала, ждала природа.
Снег выпал только в январе
На третье в ночь. Проснувшись рано,
В окно увидела Татьяна
Поутру побелевший двор,
Куртины, кровли и забор,
На стеклах легкие узоры,
Деревья в зимнем серебре,
Сорок веселых на дворе
И мягко устланные горы
Зимы блистательным ковром.
Все ярко, все бело кругом.

II
Зима!.. Крестьянин, торжествуя,
На дровнях обновляет путь;
Его лошадка, снег почуя,
Плетется рысью как-нибудь;
Бразды пушистые взрывая,
Летит кибитка удалая;
Ямщик сидит на облучке
В тулупе, в красном кушаке.
Вот бегает дворовый мальчик,
В салазки жучку посадив,
Себя в коня преобразив;
Шалун уж заморозил пальчик:
Ему и больно и смешно,
А мать грозит ему в окно...
\end{verbatim}

\paragraph*{Пример текстового файла students.csv.} Это текстовый файл с информацией о студентах; каждая строка --- информация об одном студенте, значения полей разделяются символом ";", их порядок фиксирован: имя, дата рождения в формате \texttt{ГГГГ.ММ.ДД}, оценки по физике, математике и английскому языку. В отличие от содержимого файла \texttt{onegin.txt}, строки в файле \texttt{students.csv} можно легко рассматривать и обрабатывать независимо по отдельности, поскольку их структура известна.
\begin{verbatim}
Ivanov;1999.02.28;3;4;4
Petrova;1999.04.03;5;5;4
Vasin;1999.03.01;4;4;5
Pirogov;1998.12.09;4;3;5
\end{verbatim}
Расширение файла \texttt{students.csv} состоит из трёх букв \texttt{csv}; это расшифровывается как "comma separated values", т.е. "значения, разделённые символом запятой" --- в качестве разделителя допустимо использовать и символ ";", смысл сохраняется.

\subsubsection*{2. Двоичные файлы}
Двоичные (бинарные) файлы --- это файлы, содержимое которых подготавливается программами и для программ, т.е. это машиночитаемый тип файлов, которые не создаются или редактируются человеком без использования специальных программ. Все программы, которые позволяют обработку двоичных файлов, должны следовать одной и той же спецификации --- описанию формата данных в файле.

Пример двоичных файлов --- файлы документов \texttt{.doc} или \texttt{.docx}, доступные для редактирования только в специальных приложениях, умеющих работать с данным форматом. Если открыть \texttt{doc}-файл в текстовом редакторе "Блокнот", то можно увидеть множество непонятных символов, поскольку "Блокнот" пытается распознать двоичный файл как текстовый, каждый байт которого кодирует некоторый символ. Однако если открыть \texttt{doc}-файл в специальном редакторе, например, Microsoft Word или Open Office Writer, то содержимое файла будет вполне понятно.

В Delphi двоичные файлы принято подразделять на нетипизированные и типизированные. При работе с нетипизированными файлами программа выполняет чтение и запись по одному байту, и сама определяет, какое содержимое и как какие байты кодируют --- нетипизированный файл может состоять из компонент разного типа. При работе с типизированными файлами в коде программы указывается тип компонент (один из стандартных или объявленных в \texttt{type} типов данных Delphi), из которых состоит файл, и когда программа выполняет запись компоненты заданного типа автоматически перекодируются в последовательность байт, которые записываются в файл. Чтение из типизированного файла также выполняется отдельными компонентами: программа считывает последовательность байт, соответствующих одной компоненте, и преобразует их в заданный тип. Например, если файл состоит из целых чисел, т.е. компонент типа Integer, программа считывает и записывает данные по 4 байта, поскольку одно целое число \texttt{Integer} в Delphi занимает ровно 4 байта. Если в типизированный файл из Integer-компонент записать 3 целых числа, тогда итоговый размер файла будет ровно 12 байт (3 раза по 4 байта).

\clearpage
\section*{Типизированные файлы}
\subsection*{Схема работы с типизированным файлом}
\begin{enumerate}
\item Связать файловую переменную и имя файла, который программа будет обрабатывать (\texttt{AssignFile});
\item Открыть существующий файл для чтения/записи (\texttt{Reset}) или создать новый файл/перезаписать существующий (\texttt{ReWrite});
\item Выполнение операций чтения и/или записи компонент файла (\texttt{Read/Write});
\item Закрытие файла для фиксации всех изменений на диске (\texttt{CloseFile}).
\end{enumerate}

Порядок этапов работы с файлом не может быть изменён: нельзя записать данные в файл, который не был открыт; и нельзя закрывать файл, если, например, ещё предполагается чтение компонент. Для каждого этапа в скобках указаны названия соответствующих стандартных процедур, используемых при обработке типизированных файлов.

\subsection*{Объявление файловой переменной}
Прежде, чем работать с файлом, в программе необходимо объявить файловую переменную и описать тип компонент, из которых состоит файл.

\textbf{Пример 1.} Объявление файловой переменной для работы с файлом целых чисел (компоненты файла имеют тип Integer), вариант 1.
\begin{verbatim}
var
  F: file of Integer;
\end{verbatim}

\textbf{Пример 2.} Объявление файловой переменной для работы с файлом целых чисел, вариант 2.
\begin{verbatim}
type
  TNumbersFile = file of Integer;
var
  F: TNumbersFile;
\end{verbatim}

\textbf{Пример 3.} Объявление файловой переменной для работы с файлом записей.
\begin{verbatim}
type
  TMark = (Udovl=3, Hor=4, Otl=5);
  TStudent = record
    Name: String[100];
    Date: TDateTime;
    Physics, Math, English: TMark;
  end;
var
  StudentsFile: file of TStudent;
\end{verbatim}

\subsection*{Процедура AssignFile}
После того, как файловая переменная объявленная, её можно использовать в программе для работы с файлами. Первое, что требуется сделать, --- указать имя файла, который программа будет обрабатывать. Это выполняет стандартная процедура \texttt{AssignFile}.
\begin{verbatim}
procedure AssignFile ( var FileHandle : File; const FileName : string ) ;
\end{verbatim}

\begin{itemize}
\item \texttt{FileHandle} --- файловая переменная (например, StudentsFile, как в примере 3 объявления файловой переменной);
\item \texttt{FileName} --- имя файла, с которым будет работать программа (можно указать \texttt{String}-переменную, а можно просто в коде написать имя файла как строку: \texttt{AssignFile(F, 'numbers.dat')}).
\end{itemize}

\subsection*{Создание или перезапись типизированного файла --- ReWrite}
Процедура \texttt{ReWrite} вызывается для создания нового файла, или когда необходимо перезаписать существующий файл (во втором случае вызов \texttt{ReWrite} очищает существующий файл целиком).
\begin{verbatim}
procedure Rewrite ( var FileHandle : File ) ;
\end{verbatim}
\begin{itemize}
\item \texttt{FileHandle} --- файловая переменная.
\end{itemize}


\subsection*{Процедура Reset}
Вызов \texttt{Reset} открывает ранее созданный файл для чтения/записи. \texttt{Reset} нельзя вызывать, если файл, с которым связана файловая переменная, не существует; иначе будет вызвано исключение, и программа завершится аварийно.
\begin{verbatim}
procedure Reset ( var FileHandle : File ) ;
\end{verbatim}
\begin{itemize}
\item \texttt{FileHandle} --- файловая переменная.
\end{itemize}

Указатель чтения/записи файловой переменной сразу после вызова \texttt{Reset} устанавливается в начало типизированного файла перед первой компонентой.


\subsection*{Процедура Re}
\texttt{}
\begin{verbatim}

\end{verbatim}
\begin{itemize}
\item \texttt{FileHandle} --- файловая переменная
\end{itemize}


\subsection*{}
\texttt{}
\begin{verbatim}

\end{verbatim}
\begin{itemize}
\item \texttt{FileHandle} --- файловая переменная
\end{itemize}


\subsection*{Процедура Re}
\texttt{}
\begin{verbatim}

\end{verbatim}
\begin{itemize}
\item \texttt{FileHandle} --- файловая переменная
\end{itemize}





\end{document}
