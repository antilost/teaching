\documentclass[12pt,a4paper]{report}
%\documentclass[12pt,a5paper,landscape]{report}

\usepackage[left=1.5cm,right=1.5cm,top=1.5cm,bottom=1.5cm,bindingoffset=0cm,pdftex]{geometry}
\usepackage[utf8]{inputenc}
\usepackage{indentfirst}

\usepackage{color}
\usepackage{listings}
\usepackage{amsmath}
\usepackage{textcomp} % Для значка градуса \textdegree
\usepackage{amssymb}  % Для значка треугольника
\usepackage{amsmath}

\usepackage[russian]{babel}

\begin{document}
\parindent=1cm
\pagestyle{empty}


%\clearpage
%\subsection*{Вариант 1}
%\paragraph*{Задача №1}
%Вычисление метрик по значениям матрицы.
%\paragraph*{Задача №2}
%Формирование матрицы из чисел Фибоначчи.


\clearpage
\paragraph*{Задача №1.1}
Задана квадратная матрица $A_{n \times n}, n \le 20$, состоящая из целых чисел. Найдите сумму элементов, которые расположены на главной или побочной диагонали. Центральный элемент, стоящий на пересечении диагоналей матрицы, учитывать в сумме только один раз.

\paragraph*{Задача №1.2}
Задана матрица $A_{m \times n}, m, n \le 20$, состоящая из целых чисел; и два натуральных числа $p, q, 1 \le p \le m, 1 \le q \le m, p \ne q$ -- номера двух строк в матрице $A_{m \times n}$. Поменяйте местами строку с номером $p$ и строку с номером $q$.

\paragraph*{Задача №1.3}
Задана матрица $A_{m \times n}, m, n \le 20$, состоящая из целых чисел. Найдите номер строки, с наибольшей суммой элементов.

\paragraph*{Задача №1.4}
Задана матрица $A_{m \times n}, m, n \le 20$, состоящая из целых чисел; и два натуральных числа $g, t, 1 \le g \le n, 1 \le t \le n, g \ne t$ -- номера двух столбцов в матрице $A_{m \times n}$. Поменяйте местами столбец с номером $g$ и столбец с номером $t$.

\paragraph*{Задача №1.5}
Задана матрица $A_{m \times n}, m, n \le 20$, состоящая из целых чисел. Найдите номер столбца, с наименьшей суммой элементов.

\paragraph*{Задача №1.6}
Задана квадратная матрица $A_{n \times n}, n \le 20$, состоящая из целых чисел. Вычислите сумму элементов, расположенных выше главной диагонали (элементы, стоящие на самой диагонали, в сумму не включаются).

\paragraph*{Задача №1.7}
Задана квадратная матрица $A_{n \times n}, n \le 20$, состоящая из целых чисел. Транспонируйте матрицу $A_{n \times n}$.

\paragraph*{Задача №1.8}
Задана матрица $A_{m \times n}, m, n \le 20$, состоящая из целых чисел. Найдите номер первой строки $i$, в которой нет ни одного отрицательного числа $a_{i, j} < 0, j=1..n$.

\paragraph*{Задача №1.9}
Заданы две матрицы $A_{m \times n} \text{ и } B_{m \times n}, m, n \le 20$, состоящие из целых чисел. Вычислите матрицу-сумму $C_{m \times n} = A_{m \times n} + B_{m \times n}$.

\paragraph*{Задача №1.10}
Заданы две матрицы $A_{m \times n} \text{ и } B_{m \times n}, m, n \le 20$, состоящие из целых чисел. Вычислите матрицу-разность $D_{m \times n} = A_{m \times n} - B_{m \times n}$.

\paragraph*{Задача №1.1}
Задана квадратная матрица $A_{n \times n}, n \le 20$, состоящая из целых чисел. Найдите сумму элементов, которые расположены на главной или побочной диагонали. Центральный элемент, стоящий на пересечении диагоналей матрицы, учитывать в сумме только один раз.

\paragraph*{Задача №1.2}
Задана матрица $A_{m \times n}, m, n \le 20$, состоящая из целых чисел; и два натуральных числа $p, q, 1 \le p \le m, 1 \le q \le m, p \ne q$ -- номера двух строк в матрице $A_{m \times n}$. Поменяйте местами строку с номером $p$ и строку с номером $q$.

\paragraph*{Задача №1.4}
Задана матрица $A_{m \times n}, m, n \le 20$, состоящая из целых чисел; и два натуральных числа $g, t, 1 \le g \le n, 1 \le t \le n, g \ne t$ -- номера двух столбцов в матрице $A_{m \times n}$. Поменяйте местами столбец с номером $g$ и столбец с номером $t$.

\paragraph*{Задача №1.6}
Задана квадратная матрица $A_{n \times n}, n \le 20$, состоящая из целых чисел. Вычислите сумму элементов, расположенных выше главной диагонали (элементы, стоящие на самой диагонали, в сумму не включаются).


\clearpage
\noindent % чтобы minipage не был с красной строки
\begin{minipage}{0.75\textwidth}
\paragraph*{Задача №2.1}
Заданы натуральные числа $m, n \le 10$. Получить матрицу $B_{m \times n}$ из $m$ строк и $n$ столбцов, заполнив её первыми $nm$ числами Фибоначчи по строкам справа-налево, по столбцам сверху-вниз (справа пример такой матрицы $B_{5 \times 4}$). Формула для расчёта чисел Фибоначчи: \\
$\begin{cases}
F_0 = 0; F_1 = 1; F_i = F_{i-2} + F_{i-1}, & \text{при } i \ge 2. \\
\end{cases}$
\end{minipage}
\hfill
\begin{minipage}{0.24\textwidth}
$\begin{pmatrix}
2 & 1 & 1 & 0 \\
13 & 8 & 5 & 3 \\
89 & 55 & 34 & 21 \\
610 & 377 & 233 & 144 \\
4181 & 2584 & 1597 & 987 \\
\end{pmatrix}$
\end{minipage}
\\ \\ \\
\noindent % чтобы minipage не был с красной строки
\begin{minipage}{0.75\textwidth}
\paragraph*{Задача №2.2}
Заданы натуральные числа $m, n \le 10$. Получить матрицу $B_{m \times n}$ из $m$ строк и $n$ столбцов, заполнив её первыми $nm$ числами Фибоначчи по строкам слева-направо, по столбцам снизу-вверх (справа пример такой матрицы $B_{5 \times 4}$). Формула для расчёта чисел Фибоначчи: \\
$\begin{cases}
F_0 = 0; F_1 = 1; F_i = F_{i-2} + F_{i-1}, & \text{при } i \ge 2. \\
\end{cases}$
\end{minipage}
\hfill
\begin{minipage}{0.24\textwidth}
$\begin{pmatrix}
987 & 1597 & 2584 & 4181 \\
144 & 233 & 377 & 610 \\
21 & 34 & 55 & 89 \\
3 & 5 & 8 & 13 \\
0 & 1 & 1 & 2 \\
\end{pmatrix}$
\end{minipage}
\\ \\ \\
\noindent % чтобы minipage не был с красной строки
\begin{minipage}{0.75\textwidth}
\paragraph*{Задача №2.3}
Задано натуральное число $n \le 15$. Получить квадратную матрицу $B_{n \times n}$, заполнив все её крайние элементы по часовой стрелке первыми $4(n-1)$ числами Фибоначчи; все остальные элементы установите равными -1 (справа пример такой матрицы $B_{5 \times 5}$). Формула для расчёта чисел Фибоначчи: \\
$\begin{cases}
F_0 = 0; F_1 = 1; F_i = F_{i-2} + F_{i-1}, & \text{при } i \ge 2. \\
\end{cases}$
\end{minipage}
\hfill
\begin{minipage}{0.24\textwidth}
$\begin{pmatrix}
0 & 1 & 1 & 2 & 3 \\
610 & -1 & -1 & -1 & 5 \\
377 & -1 & -1 & -1 & 8 \\
233 & -1 & -1 & -1 & 13 \\
144 & 89 & 55 & 34 & 21 \\
\end{pmatrix}$
\end{minipage}
\\ \\ \\
\noindent % чтобы minipage не был с красной строки
\begin{minipage}{0.75\textwidth}
\paragraph*{Задача №2.1}
Заданы натуральные числа $m, n \le 10$. Получить матрицу $B_{m \times n}$ из $m$ строк и $n$ столбцов, заполнив её первыми $nm$ числами Фибоначчи по строкам справа-налево, по столбцам сверху-вниз (справа пример такой матрицы $B_{5 \times 4}$). Формула для расчёта чисел Фибоначчи: \\
$\begin{cases}
F_0 = 0; F_1 = 1; F_i = F_{i-2} + F_{i-1}, & \text{при } i \ge 2. \\
\end{cases}$
\end{minipage}
\hfill
\begin{minipage}{0.24\textwidth}
$\begin{pmatrix}
2 & 1 & 1 & 0 \\
13 & 8 & 5 & 3 \\
89 & 55 & 34 & 21 \\
610 & 377 & 233 & 144 \\
4181 & 2584 & 1597 & 987 \\
\end{pmatrix}$
\end{minipage}
\\ \\ \\
\noindent % чтобы minipage не был с красной строки
\begin{minipage}{0.75\textwidth}
\paragraph*{Задача №2.2}
Заданы натуральные числа $m, n \le 10$. Получить матрицу $B_{m \times n}$ из $m$ строк и $n$ столбцов, заполнив её первыми $nm$ числами Фибоначчи по строкам слева-направо, по столбцам снизу-вверх (справа пример такой матрицы $B_{5 \times 4}$). Формула для расчёта чисел Фибоначчи: \\
$\begin{cases}
F_0 = 0; F_1 = 1; F_i = F_{i-2} + F_{i-1}, & \text{при } i \ge 2. \\
\end{cases}$
\end{minipage}
\hfill
\begin{minipage}{0.24\textwidth}
$\begin{pmatrix}
987 & 1597 & 2584 & 4181 \\
144 & 233 & 377 & 610 \\
21 & 34 & 55 & 89 \\
3 & 5 & 8 & 13 \\
0 & 1 & 1 & 2 \\
\end{pmatrix}$
\end{minipage}
\\ \\ \\
\noindent % чтобы minipage не был с красной строки
\begin{minipage}{0.75\textwidth}
\paragraph*{Задача №2.3}
Задано натуральное число $n \le 15$. Получить квадратную матрицу $B_{n \times n}$, заполнив все её крайние элементы по часовой стрелке первыми $4(n-1)$ числами Фибоначчи; все остальные элементы установите равными -1 (справа пример такой матрицы $B_{5 \times 5}$). Формула для расчёта чисел Фибоначчи: \\
$\begin{cases}
F_0 = 0; F_1 = 1; F_i = F_{i-2} + F_{i-1}, & \text{при } i \ge 2. \\
\end{cases}$
\end{minipage}
\hfill
\begin{minipage}{0.24\textwidth}
$\begin{pmatrix}
0 & 1 & 1 & 2 & 3 \\
610 & -1 & -1 & -1 & 5 \\
377 & -1 & -1 & -1 & 8 \\
233 & -1 & -1 & -1 & 13 \\
144 & 89 & 55 & 34 & 21 \\
\end{pmatrix}$
\end{minipage}
\\ \\ \\
\noindent % чтобы minipage не был с красной строки
\begin{minipage}{0.75\textwidth}
\paragraph*{Задача №2.3}
Задано натуральное число $n \le 15$. Получить квадратную матрицу $B_{n \times n}$, заполнив все её крайние элементы по часовой стрелке первыми $4(n-1)$ числами Фибоначчи; все остальные элементы установите равными -1 (справа пример такой матрицы $B_{5 \times 5}$). Формула для расчёта чисел Фибоначчи: \\
$\begin{cases}
F_0 = 0; F_1 = 1; F_i = F_{i-2} + F_{i-1}, & \text{при } i \ge 2. \\
\end{cases}$
\end{minipage}
\hfill
\begin{minipage}{0.24\textwidth}
$\begin{pmatrix}
0 & 1 & 1 & 2 & 3 \\
610 & -1 & -1 & -1 & 5 \\
377 & -1 & -1 & -1 & 8 \\
233 & -1 & -1 & -1 & 13 \\
144 & 89 & 55 & 34 & 21 \\
\end{pmatrix}$
\end{minipage}


\end{document}