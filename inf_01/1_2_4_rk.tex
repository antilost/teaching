\documentclass[12pt,a5paper,landscape]{report}

\usepackage[left=1cm,right=1cm,top=1.5cm,bottom=1cm,bindingoffset=0cm,pdftex]{geometry}
\usepackage[utf8]{inputenc}
\usepackage{indentfirst}

\usepackage{color}
\usepackage{listings}
\usepackage{amsmath}
\usepackage{textcomp} % Для значка градуса \textdegree
\usepackage{amssymb}  % Для значка треугольника
\usepackage{amsmath}

\usepackage[russian]{babel}

\begin{document}
\parindent=1cm
\pagestyle{empty}

\clearpage
\section*{Рубежный контроль №2}
\subsection*{Вариант 1}
\subsubsection*{Задача №1}
Задано натуральное число $n \le 15$, массив целых чисел $B_n$ и целое число $p$. Найдите сумму элементов массива $B_n$, которые меньше числа $p$.
\subsubsection*{Задача №2}
Заданы натуральные числа $n, m \le 10$ и матрица целых чисел $A_{n \times m}$ из $n$ строк и $m$ столбцов. Найдите номер строки с наименьшей суммой элементов.
\subsubsection*{Задача №3}
Задано натуральное число $n \le 20$ и массив целых чисел $C_n$. В массиве $C_n$ найдите наименьшее значение и удалите все элементы, равные этому значению.


\clearpage
\section*{Рубежный контроль №2}
\subsection*{Вариант 2}
\subsubsection*{Задача №1}
Задано натуральное число $n \le 15$ и массив целых чисел $B_n$. Найдите индекс (порядковый номер) элемента в массиве $B_n$ с наибольшим значением.
\subsubsection*{Задача №2}
Задано натуральное число $n \le 10$ и квадратная матрица целых чисел $A_{n \times n}$. Вычислить сумму элементов, расположенных на побочной диагонали матрицы.
\subsubsection*{Задача №3}
Задано натуральное число $n \le 20$ и массив целых чисел $C_n$. Переставьте в начало массива $C_n$ все элементы, равные $0$, сохранив порядок следования остальных элементов.


\clearpage
\section*{Рубежный контроль №2}
\subsection*{Вариант 3}
\subsubsection*{Задача №1}
Задано натуральное число $n \le 20$. Сформируйте массив действительных чисел $B_n$, в котором $b_i = 4 \cos { \frac {2\pi(i-1)} {n} }, 1 \le i \le n$.
\subsubsection*{Задача №2}
Заданы натуральные числа $n, m \le 10$ и матрица целых чисел $A_{n \times m}$ из $n$ строк и $m$ столбцов. Найдите номер столбца с наименьшей суммой элементов.
\subsubsection*{Задача №3}
Задано натуральное число $n \le 20$, массив целых чисел $C_n$ и массив действительных чисел $D_n$ (оба массива одинаковой длины). Найдите минимум $\min({ c_1 d_n, c_2 d_{n-1}, \ldots, c_n d_1 })$.


\clearpage
\section*{Рубежный контроль №2}
\subsection*{Вариант 4}
\subsubsection*{Задача №1}
Задано натуральное число $n \le 15$ и массив целых чисел $B_n$. Вывести \texttt{TRUE}, если все элементы массива $B_n$ отрицательные ($b_i < 0, \text{для любого }i: 1 \le i \le n$), и \texttt{FALSE}, иначе.
\subsubsection*{Задача №2}
Задано натуральное число $n \le 10$ и квадратная матрица целых чисел $A_{n \times n}$. Вычислить сумму элементов, расположенных на главной диагонали матрицы.
\subsubsection*{Задача №3}
Задано натуральное число $n \le 20$ и массив целых чисел $C_n$. В массиве $C_n$ поменяйте местами наименьшее и наибольшее значения.


\clearpage
\section*{Рубежный контроль №2}
\subsection*{Вариант 5}
\subsubsection*{Задача №1}
Задано натуральное число $n \le 20$. Сформируйте массив действительных чисел $B_n$, в котором $b_i = 3 \sin^2 { \frac {2\pi(i-1)} {n} }, 1 \le i \le n$.
\subsubsection*{Задача №2}
Заданы натуральные числа $n, m \le 10$ и матрица целых чисел $A_{n \times m}$ из $n$ строк и $m$ столбцов. Найдите номер столбца с наибольшей суммой элементов.
\subsubsection*{Задача №3}
Задано натуральное число $n \le 20$ и массив целых чисел $C_n$. В массиве $C_n$ найдите наименьшее значение и удалите все элементы, равные этому значению.


\clearpage
\section*{Рубежный контроль №2}
\subsection*{Вариант 6}
\subsubsection*{Задача №1}
Задано натуральное число $n \le 15$ и массив целых чисел $B_n$. Вычислите среднее арифметическое элементов массива $B_n$.
\subsubsection*{Задача №2}
Заданы натуральные числа $n, m \le 10$, две матрицы целых чисел $A_{n \times m}, B_{n \times m}$ из $n$ строк и $m$ столбцов. Вычислите матрицу-сумму $C = A + B$.
\subsubsection*{Задача №3}
Задано натуральное число $n \le 20$ и массив целых чисел $C_n$. Переставьте в начало массива $C_n$ все элементы, равные $0$, сохранив порядок следования остальных элементов.


\clearpage
\section*{Рубежный контроль №2}
\subsection*{Вариант 7}
\subsubsection*{Задача №1}
Задано натуральное число $n \le 20$. Сформируйте массив действительных чисел $B_n$, в котором $b_i = \sqrt { \frac {100 (i-1)} {n} }, 1 \le i \le n$.
\subsubsection*{Задача №2}
Заданы натуральные числа $n, m \le 10$ и матрица целых чисел $A_{n \times m}$ из $n$ строк и $m$ столбцов. Найдите номер строки с наибольшей суммой элементов.
\subsubsection*{Задача №3}
Задано натуральное число $n \le 20$, массив целых чисел $C_n$ и массив действительных чисел $D_n$ (оба массива одинаковой длины). Найдите минимум $\min({ c_1 d_n, c_2 d_{n-1}, \ldots, c_n d_1 })$.


\clearpage
\section*{Рубежный контроль №2}
\subsection*{Вариант 8}
\subsubsection*{Задача №1}
Задано натуральное число $n \le 15$ и массив целых чисел $B_n$. Вывести \texttt{TRUE}, если в массиве есть хотя бы один нечётный элемент, и \texttt{FALSE}, иначе.
\subsubsection*{Задача №2}
Заданы натуральные числа $n, m \le 10$, две матрицы целых чисел $A_{n \times m}, B_{n \times m}$ из $n$ строк и $m$ столбцов. Вычислите матрицу-разность $C = A - B$.
\subsubsection*{Задача №3}
Задано натуральное число $n \le 20$. Сформируйте одномерный массив из первых $n$ чисел Фибоначчи $0, 1, 1, 2, 3, 5, 8, 13, \ldots$.


\clearpage
\section*{Рубежный контроль №2}
\subsection*{Вариант 9}
\subsubsection*{Задача №1}
Задано натуральное число $n \le 20$. Сформируйте массив действительных чисел $B_n$, в котором $b_i = 0.5 \cos { \frac {2 \pi (i-1)} {n} }, 1 \le i \le n$.
\subsubsection*{Задача №2}
Заданы натуральные числа $n, m \le 10$ и матрица целых чисел $A_{n \times m}$ из $n$ строк и $m$ столбцов. Найдите номер столбца, который содержит наибольшее для всей матрицы значение (глобальный максимум).
\subsubsection*{Задача №3}
Задано натуральное число $n \le 20$ и массив целых чисел $C_n$. В массиве $C_n$ найдите наименьшее значение и удалите все элементы, равные этому значению.


\clearpage
\section*{Рубежный контроль №2}
\subsection*{Вариант 10}
\subsubsection*{Задача №1}
Задано натуральное число $n \le 15$ и массив целых чисел $B_n$. Вычислите сумму всех элементов массива $B_n$ с нечётными индексами: $B_1+B_3+B_5+\ldots $.
\subsubsection*{Задача №2}
Заданы натуральные числа $n, m \le 10$. Сформируйте матрицу $A_{n \times m}$ из $n$ строк и $m$ столбцов, заполнив её чётными числами подряд $2, 4, 6, \ldots$ справа-налево по строкам и снизу-вверх по столбцам.
\subsubsection*{Задача №3}
Задано натуральное число $n \le 20$ и массив целых чисел $C_n$. Переставьте в начало массива $C_n$ все элементы, равные $0$, сохранив порядок следования остальных элементов.


\clearpage
\section*{Рубежный контроль №2}
\subsection*{Вариант 11}
\subsubsection*{Задача №1}
Задано натуральное число $n \le 20$. Сформируйте массив действительных чисел $B_n$, в котором $b_i = (\frac {10(i-1)} {n}-5)^2, 1 \le i \le n$.
\subsubsection*{Задача №2}
Заданы натуральные числа $n, m \le 10$ и матрица целых чисел $A_{n \times m}$ из $n$ строк и $m$ столбцов. Найдите номер строки, которая содержит наибольшее для всей матрицы значение (глобальный максимум).
\subsubsection*{Задача №3}
Задано натуральное число $n \le 20$, массив целых чисел $C_n$ и массив действительных чисел $D_n$ (оба массива одинаковой длины). Найдите минимум $\min({ c_1 d_n, c_2 d_{n-1}, \ldots, c_n d_1 })$.


\clearpage
\section*{Рубежный контроль №2}
\subsection*{Вариант 12}
\subsubsection*{Задача №1}
Задано натуральное число $n \le 15$, массив действительных чисел $B_n$ и действительное число $k$. Найдите наибольшее значение среди элементов массива $B_n$, домноженных на коэффициент $k$: $\max({k b_1, k b_2, \ldots, k b_n})$.
\subsubsection*{Задача №2}
Заданы натуральные числа $n, m \le 10$. Сформируйте матрицу $A_{n \times m}$ из $n$ строк и $m$ столбцов, заполнив её нечётными числами подряд $1, 3, 5, \ldots$ слева-направо по строкам и снизу-вверх по столбцам.
\subsubsection*{Задача №3}
Задано натуральное число $n \le 20$, массив целых чисел $C_n$ и целое число $p > 1$. Удалите из массива $C_n$ все элементы, кратные числу $p$.


\clearpage
\section*{Рубежный контроль №2}
\subsection*{Вариант 13}
\subsubsection*{Задача №1}
Задано натуральное число $n \le 20$. Сформируйте массив действительных чисел $B_n$, в котором $b_i = -2 \cos { \frac {2 \pi (i-1)} {n} }, 1 \le i \le n$.
\subsubsection*{Задача №2}
Заданы натуральные числа $n, m \le 10$ и матрица целых чисел $A_{n \times m}$ из $n$ строк и $m$ столбцов. Найдите номер столбца, который содержит наименьшее для всей матрицы значение (глобальный минимум).
\subsubsection*{Задача №3}
Задано натуральное число $n \le 20$ и массив целых чисел $C_n$. В массиве $C_n$ найдите наименьшее значение и удалите все элементы, равные этому значению.



\clearpage
\section*{Рубежный контроль №2}
\subsection*{Вариант 14}
\subsubsection*{Задача №1}
Задано натуральное число $n \le 15$ и массив целых чисел $B_n$. Найдите индекс (порядковый номер) элемента в массиве $B_n$ с наименьшим значением.
\subsubsection*{Задача №2}
Задано натуральное число $n \le 10$ и квадратная матрица целых чисел $A_{n \times n}$. Заполните нулями все элементы, расположенные на главной и побочной диагоналях матрицы.
\subsubsection*{Задача №3}
Задано натуральное число $n \le 20$ и массив целых чисел $C_n$. Переставьте в начало массива $C_n$ все элементы, равные $0$, сохранив порядок следования остальных элементов.


\clearpage
\section*{Рубежный контроль №2}
\subsection*{Вариант 15}
\subsubsection*{Задача №1}
Задано натуральное число $n \le 15$, массив целых чисел $B_n$ и целое число $p > 1$. Найдите сумму элементов массива $B_n$, которые кратны числу $p$.
\subsubsection*{Задача №2}
Заданы натуральные числа $n, m \le 10$ и матрица целых чисел $A_{n \times m}$ из $n$ строк и $m$ столбцов. Найдите номер строки, которая содержит наименьшее для всей матрицы значение (глобальный минимум).
\subsubsection*{Задача №3}
Задано натуральное число $n \le 20$, массив целых чисел $C_n$ и массив действительных чисел $D_n$ (оба массива одинаковой длины). Найдите максимум $\max({ c_1 d_n, c_2 d_{n-1}, \ldots, c_n d_1 })$.


\clearpage
\section*{Рубежный контроль №2}
\subsection*{Вариант 16}
\subsubsection*{Задача №1}
Задано натуральное число $n \le 15$, массив целых чисел $B_n$ и целое число $p$. В массиве $B_n$ замените каждый элемент $b_i > p$ на $0, i = 1..n$.
\subsubsection*{Задача №2}
Задано натуральное число $n \le 10$. Сформируйте квадратную единичную матрицу порядка $n$.
\subsubsection*{Задача №3}
Задано натуральное число $n \le 20$, массив целых чисел $C_n$ и целое число $p > 1$. Удалите из массива $C_n$ все элементы, не кратные числу $p$.


\clearpage
\section*{Рубежный контроль №2}
\subsection*{Вариант 17}
\subsubsection*{Задача №1}
Задано натуральное число $n \le 20$. Сформируйте массив действительных чисел $B_n$, в котором $b_i = (\frac {4(i-1)} {n})^2, 1 \le i \le n$.
\subsubsection*{Задача №2}
Заданы натуральные числа $n, m \le 10$ и матрица целых чисел $A_{n \times m}$ из $n$ строк и $m$ столбцов. Сформируйте одномерный массив $D_n$ из $n$ целых чисел, в котором $i$-й элемент $d_i$ будет представлять сумму элементов в $i$-й строке матрицы $A_{n \times m}$.
\subsubsection*{Задача №3}
Задано натуральное число $n \le 20$ и массив целых чисел $C_n$. В массиве $C_n$ найдите наименьшее значение и удалите все элементы, равные этому значению.


\clearpage
\section*{Рубежный контроль №2}
\subsection*{Вариант 18}
\subsubsection*{Задача №1}
Задано натуральное число $n \le 15$, массив целых чисел $B_n$ и целое число $p$. Вывести \texttt{TRUE}, если в массиве $B_n$ есть элементы, равные числу $p$, и \texttt{FALSE}, иначе.
\subsubsection*{Задача №2}
Задано натуральное число $n \le 10$ и квадратная матрица целых чисел $A_{n \times n}$. Найдите наибольшее значение, среди всех элементов, расположенных на главной диагонали матрицы $A_{n \times n}$.
\subsubsection*{Задача №3}
Задано натуральное число $n \le 20$ и массив целых чисел $C_n$. Переставьте в начало массива $C_n$ все элементы, равные $0$, сохранив порядок следования остальных элементов.


\clearpage
\section*{Рубежный контроль №2}
\subsection*{Вариант 19}
\subsubsection*{Задача №1}
Задано натуральное число $n \le 15$, массив целых чисел $B_n$ и целое число $p$. Найдите количество элементов массива $B_n$, которые не превышают число $p$.
\subsubsection*{Задача №2}
Заданы натуральные числа $n, m \le 10$ и матрица целых чисел $A_{n \times m}$ из $n$ строк и $m$ столбцов. Сформируйте одномерный массив $D_m$ из $m$ целых чисел, в котором $i$-й элемент $d_i$ будет представлять сумму элементов в $i$-м столбце матрицы $A_{n \times m}$.
\subsubsection*{Задача №3}
Задано натуральное число $n \le 20$, массив целых чисел $C_n$ и массив действительных чисел $D_n$ (оба массива одинаковой длины). Найдите максимум $\max({ c_1 d_n, c_2 d_{n-1}, \ldots, c_n d_1 })$.


\clearpage
\section*{Рубежный контроль №2}
\subsection*{Вариант 20}
\subsubsection*{Задача №1}
Задано натуральное число $n \le 15$ и массив целых чисел $B_n$. В массиве $B_n$ замените каждый элемент $b_i < 0$ на $3|b_i|, i = 1..n$.
\subsubsection*{Задача №2}
Задано натуральное число $n \le 10$ и квадратная матрица целых чисел $A_{n \times n}$. Найдите наименьшее значение, среди всех элементов, расположенных на побочной диагонали матрицы $A_{n \times n}$.
\subsubsection*{Задача №3}
Задано натуральное чётное число $n \le 20$ и массив целых чисел $B_n$. Обеспечьте выполнение равенства $B_{2j-1} > B_{2j}$ для каждой пары элементов $B_{2j-1}, B_{2j}, j=1..\frac{n}{2}$, при необходимости поменяв значения элементов между собой.


\end{document}