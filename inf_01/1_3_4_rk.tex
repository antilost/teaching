\documentclass[12pt,a5paper,landscape]{report}

\usepackage[left=1cm,right=1cm,top=1.5cm,bottom=1cm,bindingoffset=0cm,pdftex]{geometry}
\usepackage[utf8]{inputenc}
\usepackage{indentfirst}

\usepackage{color}
\usepackage{listings}
\usepackage{amsmath}
\usepackage{textcomp} % Для значка градуса \textdegree
\usepackage{amssymb}  % Для значка треугольника
\usepackage{amsmath}

\usepackage[russian]{babel}

\begin{document}
\parindent=1cm
\pagestyle{empty}

\clearpage
\section*{Рубежный контроль №3}
\subsection*{Вариант 1}
\subsubsection*{Задача №1}
Получить таблицу значений функции $y=f(x)$ для $x$, изменяющегося в диапазоне $[a; b]$ с шагом $\Delta x$ (заданы действительные числа $a, b, \Delta x, a<b, \Delta x < |b-a|$). Для вычисления $y=f(x)$ использовать подпрограмму-функцию (\texttt{function}).
\begin{equation*}
f(x) = 
\begin{cases}
  x^2, &\text{если $x \le 0$};\\
  0.5 x \ln x, &\text{если $x > 0$}.\\
\end{cases}
\end{equation*}
\subsubsection*{Задача №2}
Задана строка, в которой записаны числа в формате \texttt{<число>}, символ «;» и ноль и более пробелов. Вывести минимальное и максимальное числа. Например, для строки \verb|"12; 135;   77; 9;    11;"| результат будет 9 и 135.
\subsubsection*{Задача №3}
Ввести простое предложение, состоящее из английских слов, разделённых пробелами (между словами ровно один пробел). В конце предложения всегда стоит один из знаков «.», «!», «?». Вывести топ 10 самых часто встречающихся в предложении букв английского алфавита по убыванию частоты: вывести букву и число вхождений. Программа должна игнорировать регистр, т.е. не различать заглавные и строчные буквы.

\section*{Рубежный контроль №3}
\subsection*{Вариант 2}
\subsubsection*{Задача №1}
Задано натуральное число $n \le 15$, массив целых чисел $B_n$ и целое число $p$. Найдите сумму элементов массива $B_n$, которые меньше числа $p$. Для решения задачи использовать подпрограмму-функцию (\texttt{function}).
\subsubsection*{Задача №2}
Ввести простое предложение, состоящее из английских слов, разделённых пробелами (между словами ровно один пробел). В конце предложения всегда стоит один из знаков «.», «!», «?». Поменять местами первое и последнее слова в предложении. Первая буква в предложении должна быть заглавной и после преобразования, а все остальные -- строчными.
\subsubsection*{Задача №3}
Задано натуральное число $n \le 15$ и массив целых чисел $B_n$. Удалить в массиве $B_n$ все дубликаты чисел (для повторяющихся чисел оставить ровно одно вхождение) и упорядочить массив по убыванию. Для ввода, вывода и обработки массива использовать отдельные подпрограммы (функции \texttt{function} или процедуры \texttt{procedure}).


\section*{Рубежный контроль №3}
\subsection*{Вариант 3}
\subsubsection*{Задача №1}
Задано натуральное число $n$ и ровно $n$ действительных чисел $x$. Для каждого $x$ вычислить значение функции $y=f(x)$, используя подпрограмму-процедуру (\texttt{procedure}).
\begin{equation*}
f(x) = 
\begin{cases}
  2|x|, &\text{если $x \le 1$};\\
  \frac{1}{x-1}, &\text{если $x > 1$}.\\
\end{cases}
\end{equation*}
\subsubsection*{Задача №2}
Задано натуральное число $n \le 15$, массив целых чисел $B_n$ и заданный символ $c$. С помощью подпрограммы-функции \texttt{function} сформировать строку из чисел исходного массива $B_n$, соединённых заданным символом $c$.
\subsubsection*{Задача №3}
Ввести простое предложение, состоящее из английских слов, разделённых пробелами (между словами ровно один пробел). В конце предложения всегда стоит один из знаков «.», «!», «?». Вывести топ 10 самых часто встречающихся в предложении букв английского алфавита по убыванию частоты: вывести букву и число вхождений. Программа должна игнорировать регистр, т.е. не различать заглавные и строчные буквы.

\section*{Рубежный контроль №3}
\subsection*{Вариант 4}
\subsubsection*{Задача №1}
Задано натуральное число $n \le 15$ и массив целых чисел $B_n$. С помощью отдельной подпрограммы-функции \texttt{function} найдите среднее арифметическое элементов массива $B_n$.
\subsubsection*{Задача №2}
Задано натуральное число $n$ и множество $Q$ из $n$ различных символов (вводится пользователем). Ввести простое предложение, состоящее из английских слов, разделённых пробелами (между словами ровно один пробел). В конце предложения всегда стоит один из знаков «.», «!», «?». В предложении все подряд идущие символы из заданного множества $Q$ заменить на символ «*».
\subsubsection*{Задача №3}
Задано натуральное число $n \le 15$ и массив целых чисел $B_n$. Удалить в массиве $B_n$ все дубликаты чисел (для повторяющихся чисел оставить ровно одно вхождение) и упорядочить массив по убыванию. Для ввода, вывода и обработки массива использовать отдельные подпрограммы (функции \texttt{function} или процедуры \texttt{procedure}).

\clearpage
\section*{Рубежный контроль №3}
\subsection*{Вариант 5}
\subsubsection*{Задача №1}
Задано натуральное число $n \le 15$, массив целых чисел $B_n$ и целое число $p$. Вывести \texttt{TRUE}, если в массиве $B_n$ есть элементы, равные числу $p$, и \texttt{FALSE}, иначе. Для решения задачи использовать подпрограмму-функцию (\texttt{function}).
\subsubsection*{Задача №2}
Задана строка, в которой записаны числа в формате \texttt{<число>}, символ «;» и ноль и более пробелов. Вывести минимальное и максимальное числа. Например, для строки \verb|"12; 135;   77; 9;    11;"| результат будет 9 и 135.
\subsubsection*{Задача №3}
Ввести простое предложение, состоящее из английских слов, разделённых пробелами (между словами ровно один пробел). В конце предложения всегда стоит один из знаков «.», «!», «?». Вывести топ 10 самых часто встречающихся в предложении букв английского алфавита по убыванию частоты: вывести букву и число вхождений. Программа должна игнорировать регистр, т.е. не различать заглавные и строчные буквы.

\section*{Рубежный контроль №3}
\subsection*{Вариант 6}
\subsubsection*{Задача №1}
Ввести простое предложение, состоящее из английских слов, разделённых пробелами (между словами ровно один пробел). В конце предложения всегда стоит один из знаков «.», «!», «?». Перевести в верхний регистр все гласные буквы предложения (каждую строчную гласную букву заменить соответствующей ей заглавной).
\subsubsection*{Задача №2}
Задано натуральное число $n \le 10$ и два массива целых чисел $A_n$ и $B_n$. С помощью подпрограммы-функции \texttt{function} получить новый массив $C_{2n}$, в котором чередуются элементы из массива $A_n$ и $B_n$. $C_{2n} = a_1, b_1, a_2, b_2, \ldots, a_{n-1}, b_{n-1}, a_n, b_n$.  
\subsubsection*{Задача №3}
Задано натуральное число $n \le 15$ и массив целых чисел $B_n$. Удалить в массиве $B_n$ все дубликаты чисел (для повторяющихся чисел оставить ровно одно вхождение) и упорядочить массив по убыванию. Для ввода, вывода и обработки массива использовать отдельные подпрограммы (функции \texttt{function} или процедуры \texttt{procedure}).


\section*{Рубежный контроль №3}
\subsection*{Вариант 7}
\subsubsection*{Задача №1}
Задано натуральное число $n \le 15$, массив действительных чисел $B_n$ и действительное число $k$. С помощью подпрограммы-функции \texttt{function} найдите наибольшее значение среди элементов массива $B_n$, домноженных на коэффициент $k$: $\max({k b_1, k b_2, \ldots, k b_n})$.
\subsubsection*{Задача №2}
Задано натуральное число $n \le 15$, массив целых чисел $B_n$ и заданный символ $c$. С помощью подпрограммы-функции \texttt{function} сформировать строку из чисел исходного массива $B_n$, соединённых заданным символом $c$.
\subsubsection*{Задача №3}
Ввести простое предложение, состоящее из английских слов, разделённых пробелами (между словами ровно один пробел). В конце предложения всегда стоит один из знаков «.», «!», «?». Вывести топ 10 самых часто встречающихся в предложении букв английского алфавита по убыванию частоты: вывести букву и число вхождений. Программа должна игнорировать регистр, т.е. не различать заглавные и строчные буквы.

\section*{Рубежный контроль №3}
\subsection*{Вариант 8}
\subsubsection*{Задача №1}
Ввести простое предложение, состоящее из английских слов, разделённых пробелами (между словами ровно один пробел). В конце предложения всегда стоит один из знаков «.», «!», «?». Найдите количество различных строчных гласных букв в словах предложения.
\subsubsection*{Задача №2}
Заданы шесть действительных чисел $x_A, y_A, x_B, y_B, x_C, y_C$ -- координаты вершин треугольника $\Delta ABC$. По формуле Герона найдите площадь треугольника $S_{\Delta ABC} = \sqrt { p (p-a) (p-b) (p-c)}$, где $p = \frac{P}{2}$ -- полупериметр треугольника, а $a,b,c$ -- длины сторон AB, BC и AC. Для вычисления площади и периметра треугольника используйте подпрограммы-функции \texttt{function}.
\subsubsection*{Задача №3}
Задано натуральное число $n \le 15$ и массив целых чисел $B_n$. Удалить в массиве $B_n$ все дубликаты чисел (для повторяющихся чисел оставить ровно одно вхождение) и упорядочить массив по убыванию. Для ввода, вывода и обработки массива использовать отдельные подпрограммы (функции \texttt{function} или процедуры \texttt{procedure}).

\section*{Рубежный контроль №3}
\subsection*{Вариант 9}
\subsubsection*{Задача №1}
Задано натуральное число $n \le 15$ и массив целых чисел $B_n$. Найдите индекс (порядковый номер) элемента в массиве $B_n$ с наибольшим значением. Для решения задачи использовать подпрограмму-функцию (\texttt{function}).
\subsubsection*{Задача №2}
Задано натуральное число $n$ и множество $Q$ из $n$ различных символов (вводится пользователем). Ввести простое предложение, состоящее из английских слов, разделённых пробелами (между словами ровно один пробел). В конце предложения всегда стоит один из знаков «.», «!», «?». В предложении все подряд идущие символы из заданного множества $Q$ заменить на символ «*».
\subsubsection*{Задача №3}
Ввести простое предложение, состоящее из английских слов, разделённых пробелами (между словами ровно один пробел). В конце предложения всегда стоит один из знаков «.», «!», «?». Вывести топ 10 самых часто встречающихся в предложении букв английского алфавита по убыванию частоты: вывести букву и число вхождений. Программа должна игнорировать регистр, т.е. не различать заглавные и строчные буквы.

\section*{Рубежный контроль №3}
\subsection*{Вариант 10}
\subsubsection*{Задача №1}
Ввести простое предложение, состоящее из английских слов, разделённых пробелами (между словами ровно один пробел). В конце предложения всегда стоит один из знаков «.», «!», «?». Найти количество слов предложения, каждое из которых   включает в себя заданный символ $c$.
\subsubsection*{Задача №2}
Заданы шесть действительных чисел $x_A, y_A, x_B, y_B, x_C, y_C$ -- координаты вершин треугольника $\Delta ABC$. По формуле Герона найдите площадь треугольника $S_{\Delta ABC} = \sqrt { p (p-a) (p-b) (p-c)}$, где $p = \frac{P}{2}$ -- полупериметр треугольника, а $a,b,c$ -- длины сторон AB, BC и AC. Для вычисления площади и периметра треугольника используйте подпрограммы-функции \texttt{function}.
\subsubsection*{Задача №3}
Задано натуральное число $n \le 15$ и массив целых чисел $B_n$. Удалить в массиве $B_n$ все дубликаты чисел (для повторяющихся чисел оставить ровно одно вхождение) и упорядочить массив по убыванию. Для ввода, вывода и обработки массива использовать отдельные подпрограммы (функции \texttt{function} или процедуры \texttt{procedure}).





\end{document}