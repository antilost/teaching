\documentclass[12pt,a4paper]{report}
\usepackage[left=1cm,right=1cm,top=1cm,bottom=1.5cm,bindingoffset=0cm,pdftex]{geometry}
\usepackage[utf8]{inputenc}
\usepackage[russian]{babel}
\usepackage{indentfirst}
\usepackage{color}
\usepackage{listings}
\usepackage{amsmath}
\usepackage{textcomp} % Для значка градуса \textdegree
\usepackage{amssymb}  % Для значка треугольника
\usepackage{amsmath}
\begin{document}
\parindent=1cm
\pagestyle{empty}
\lstset{ language=Pascal, basicstyle=\small\ttfamily, numbers=left, numberstyle=\tiny, stepnumber=1, numbersep=5pt, extendedchars=\true, showstringspaces=false, breakatwhitespace=true, frame=single, keepspaces=true }


\section*{Работа с датой и временем}
Для представления даты и времени в Delphi используется тип данных \texttt{TDateTime}.
Тип \texttt{TDateTime} сохраняется как действительная переменная \texttt{Double}, с датой как целая часть, а время как дробная. Дата сохраняется как число дней с 30 декабря 1899. Не понятно, почему не 31 декабря. Дата 01 января 1900 имеет значение 2. 
Поскольку \texttt{TDateTime} фактически является Double, то вы можете выполнять над ним вычисления, как будто это было число. Это очень полезно для вычислений типа разницы между двумя датами.

Для обработки переменных типа \texttt{TDateTime} существует ряд перечисленных ниже функций и процедур из стандартной библиотеки \texttt{SysUtils}. Для использования таких функций и процедур в программе необходимо указать использование библиотеки \texttt{SysUtils} вот так:
\begin{verbatim}
program ExampleUsesSysUtils;

{$APPTYPE CONSOLE}

uses
  SysUtils;

...
\end{verbatim}

\subsection*{Процедуры и функции преобразования дат и времени}
\begin{itemize}
\item Функция \texttt{function Now(): TDateTime} возвращает текущую дату и время.
\item Функция \texttt{function Date(): TDateTime} возвращает текущую дату.
\item Функция \texttt{function Time(): TDateTime} возвращает текущее время.
\end{itemize}
\subsection*{Работа с составляющими даты и времени}
Получение и изменение отдельных составляющих даты и времени: год, месяц, число, день недели, часы, минуты, секунды и миллисекунды.
\begin{itemize}
\item Функция \texttt{function DayOfWeek(Date: TDateTime): Integer} возвращает текущий номер дня недели: 1 - воскресенье, 7 - суббота.
\item Процедура \texttt{procedure DecodeDate(Date: TDateTime; var Year, Month, Day: Word)} разбивает дату Date на год - Year, месяц - Month и день - Day.
\item Процедура \texttt{procedure DecodeTime(Time: TDateTime; var Hour, Min, Sec, MSec: Word)} разбивает время Time на час - Hour, минуты - Min, секунды - Sec и миллисекунды - MSec.
\item Функция \texttt{function EncodeDate(Year, Month, Day: Word): TDateTime} объединяет год - Year, месяц - Month и день - Day в значение типа TDateTime.
\item Функция \texttt{function EncodeTime(Hour, Min, Sec, MSec: Word): TDateTime} объединяет час - Hour, минуты - Min, секунды - Sec и миллисекунды - MSec в значение типа TDateTime.
\end{itemize}
\subsection*{Перевод даты и времени в строку}
\begin{itemize}
\item Функция \texttt{function DateTimeToStr(DateTime: TDateTime): String} преобразует дату и время DateTime в строку.
\item Функция \texttt{function DateToStr(Date: TDateTime): String} преобразует дату Date в строку.
\item Функция \texttt{function TimeToStr(Time: TDateTime): String} преобразует время Time в строку.
\end{itemize}

\paragraph*{Пример.} Дата и время.
\begin{verbatim}
program ExampleDateTime;

{$APPTYPE CONSOLE}

uses
  SysUtils;

type
  TStudent = record
    Name: String[20];
    YOB: TDateTime
  end;
  TA = array[1..10] of TStudent;
var
  S: TA;
  P: String;
begin
  S[1].Name := 'Vasya Ivanov';
  S[1].YOB := EncodeDate(2000, 11, 22);
  P := S[1].Name + ' ' + DateToStr(S[1].YOB);
  WriteLn(P); // Vasya Ivanov 22.11.2000
  ReadLn;
end.
\end{verbatim}


\section*{Записи (record)}
\section*{Объявление записей}

\clearpage




\subsection*{Вариант 1}
Написать программу для обработки массива записей \texttt{TRecords} с информацией о студентах: имя (строка, не более 100 символов), дата рождения (год, месяц и день) и оценки по предметам: физика, математика, английский язык.
Формат записей в блоке \texttt{type} можно объявить следующим образом:
\begin{verbatim}
TMark = (Udovl=3, Hor=4, Otl=5);
TStudent = record
  Name: String[100];
  Date: TDateTime;
  Physics, Math, English: TMark;
end;
TRecords = array of TStudent;
\end{verbatim}

Запущенная программа должна выводить меню с доступными действиями; по выбору пользователя должна быть вызвана подпрограмма, реализующая выбранное действие (каждое действие -- отдельная подпрограмма: процедура \texttt{procedure} или функция \texttt{function}). В программе должны быть реализованы обязательные действия (ввод и вывод массива записей и завершение программы) и как можно большее количество дополнительных действий. 

\subsubsection*{Обязательные действия}
\begin{itemize}
\item ввести количество записей $n > 0$ и заполнить массив информацией о $n$ студентах (если в массиве до этого были записи, то они должны быть перезаписаны);
\item вывести все записи массива в удобном для пользователя формате (с указанием полей перед выводимыми значениями и порядковых номеров);
\item завершение программы.
\end{itemize}

\subsubsection*{Дополнительные действия}
\begin{enumerate}
\item ввести данные новой записи и добавить её в конец массива после последнего элемента;
\item найти и вывести информацию о студенте по заданному имени \texttt{Name}: если студента с заданным именем нет в массиве записей, то вывести соответствующее сообщение;
\item удалить запись с заданным порядковым номером $i, 0 \le i \le n-1$, где $n$ -- начальная длина массива записей (можно получить используя функцию \texttt{Length});
\item отсортировать записи по возрастанию даты рождения \texttt{Date}.
\end{enumerate}



\clearpage
\subsection*{Вариант 2}
Написать программу для обработки массива записей \texttt{TRecords} с информацией о студентах: имя (строка, не более 100 символов), дата рождения (год, месяц и день) и оценки по предметам: физика, математика, английский язык.
Формат записей в блоке \texttt{type} можно объявить следующим образом:
\begin{verbatim}
TMark = (Udovl=3, Hor=4, Otl=5);
TStudent = record
  Name: String[100];
  Date: TDateTime;
  Physics, Math, English: TMark;
end;
TRecords = array of TStudent;
\end{verbatim}

Запущенная программа должна выводить меню с доступными действиями; по выбору пользователя должна быть вызвана подпрограмма, реализующая выбранное действие (каждое действие -- отдельная подпрограмма: процедура \texttt{procedure} или функция \texttt{function}). В программе должны быть реализованы обязательные действия (ввод и вывод массива записей и завершение программы) и как можно большее количество дополнительных действий. 

\subsubsection*{Обязательные действия}
\begin{itemize}
\item ввести количество записей $n > 0$ и заполнить массив информацией о $n$ студентах (если в массиве до этого были записи, то они должны быть перезаписаны);
\item вывести все записи массива в удобном для пользователя формате (с указанием полей перед выводимыми значениями и порядковых номеров);
\item завершение программы.
\end{itemize}

\subsubsection*{Дополнительные действия}
В программе возраст студентов вычислять относительно даты запуска программы.
\begin{enumerate}
\item ввести данные новой записи и добавить её в начало массива перед первым элементом;
\item найти количество студентов младше 18 лет;
\item вывести все записи построчно в формате: порядковый номер (индекс элемента массива), имя студента и число полных лет с даты рождения;
\item отсортировать записи по убыванию даты рождения \texttt{Date}.
\end{enumerate}

\clearpage
\subsection*{Вариант 3}
Написать программу для обработки массива записей \texttt{TRecords} с информацией о студентах: имя (строка, не более 100 символов), дата рождения (год, месяц и день) и оценки по предметам: физика, математика, английский язык.
Формат записей в блоке \texttt{type} можно объявить следующим образом:
\begin{verbatim}
TMark = (Udovl=3, Hor=4, Otl=5);
TStudent = record
  Name: String[100];
  Date: TDateTime;
  Physics, Math, English: TMark;
end;
TRecords = array of TStudent;
\end{verbatim}

Запущенная программа должна выводить меню с доступными действиями; по выбору пользователя должна быть вызвана подпрограмма, реализующая выбранное действие (каждое действие -- отдельная подпрограмма: процедура \texttt{procedure} или функция \texttt{function}). В программе должны быть реализованы обязательные действия (ввод и вывод массива записей и завершение программы) и как можно большее количество дополнительных действий. 

\subsubsection*{Обязательные действия}
\begin{itemize}
\item ввести количество записей $n > 0$ и заполнить массив информацией о $n$ студентах (если в массиве до этого были записи, то они должны быть перезаписаны);
\item вывести все записи массива в удобном для пользователя формате (с указанием полей перед выводимыми значениями и порядковых номеров);
\item завершение программы.
\end{itemize}

\subsubsection*{Дополнительные действия}
\begin{enumerate}
\item ввести данные новой записи и добавить её по указанной позиции $i, 0 \le i \le n-1$, где $n$ -- начальная длина массива записей (можно получить используя функцию \texttt{Length});
\item найти количество студентов, которые не имеют оценки "Удовлетворительно" ни по одному из предметов;
\item вывести все записи построчно в формате: порядковый номер (индекс элемента массива), имя студента \texttt{Name} и средний балл на основании трёх оценок (физика, математика, английский язык);
\item отсортировать записи по убыванию среднего балла.
\end{enumerate}


\clearpage
\subsection*{Вариант 4}
Написать программу для обработки массива записей \texttt{TRecords} с информацией о студентах: имя (строка, не более 100 символов), дата рождения (год, месяц и день) и оценки по предметам: физика, математика, английский язык.
Формат записей в блоке \texttt{type} можно объявить следующим образом:
\begin{verbatim}
TMark = (Udovl=3, Hor=4, Otl=5);
TStudent = record
  Name: String[100];
  Date: TDateTime;
  Physics, Math, English: TMark;
end;
TRecords = array of TStudent;
\end{verbatim}

Запущенная программа должна выводить меню с доступными действиями; по выбору пользователя должна быть вызвана подпрограмма, реализующая выбранное действие (каждое действие -- отдельная подпрограмма: процедура \texttt{procedure} или функция \texttt{function}). В программе должны быть реализованы обязательные действия (ввод и вывод массива записей и завершение программы) и как можно большее количество дополнительных действий. 

\subsubsection*{Обязательные действия}
\begin{itemize}
\item ввести количество записей $n > 0$ и заполнить массив информацией о $n$ студентах (если в массиве до этого были записи, то они должны быть перезаписаны);
\item вывести все записи массива в удобном для пользователя формате (с указанием полей перед выводимыми значениями и порядковых номеров);
\item завершение программы.
\end{itemize}

\subsubsection*{Дополнительные действия}
\begin{enumerate}
\item ввести количество записей $m > 0$ и добавить в конец массива информацию о $m$ студентах (в конец массива дописывается $m$ новых записей);
\item удалить первую запись в массиве (элемент с индексом 0);
\item вывести средний балл по указанному предмету среди всех записей массива (вводится название предмета);
\item отсортировать записи по полю \texttt{Name} в лексикографическом порядке по алфавиту от "A" до "z";
\end{enumerate}

\end{document}
