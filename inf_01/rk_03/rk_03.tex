\documentclass[12pt,a4paper]{report}

\usepackage[left=2cm,right=2cm,top=3cm,bottom=3cm,bindingoffset=0cm,pdftex,landscape]{geometry}
\usepackage[utf8]{inputenc}
\usepackage[russian]{babel}
\usepackage{indentfirst}

\usepackage{color}
\usepackage{listings}
\usepackage{amsmath}
\usepackage{textcomp} % Для значка градуса
\usepackage{amssymb}  % Для значка треугольника

\begin{document}
\parindent=1cm
\pagestyle{empty}

\lstset{ language=Pascal, basicstyle=\small\ttfamily, numbers=left, numberstyle=\tiny, stepnumber=1, numbersep=5pt, extendedchars=\true, showstringspaces=false, breakatwhitespace=true, frame=single, keepspaces=true }

\clearpage
\section*{Рубежный контроль №3}
\subsection*{Вариант №1}
\paragraph*{1.1.}
Задана строка, состоящая из слов, разделённых пробелами. Заменить каждый пробел на символ подчёркивания.
\begin{verbatim}
pen  pineapple apple   pen
pen__pineapple_apple___pen
\end{verbatim}
\paragraph*{1.2.}
Задан массив записей (\texttt{record}) с информацией о студентах (всего $ 1 \le N \le 10 $ записей). Каждый студент имеет имя, фамилию и количество набранных за контрольную работу баллов (от 0 до 20).
Удалить из массива всех студентов, которые набрали за работу менее 12 баллов.
\paragraph*{1.3.}
Написать собственную реализацию функции \texttt{Copy}. Cравнить результаты работы стандартного и своего вариантов процедуры для произвольной строки.
\begin{verbatim}
function MyCopy ( Source : String; StartChar, Count : Integer ) : String;
var
...
begin
...
end;
\end{verbatim}
При написании своей функции работать со строкой как с массивом символов (с доступом по индексу, например, \texttt{S[3]}). Также можно использовать стандартные функции:
\begin{enumerate}
\item \texttt{Length(S)} --- возвращает текущую длину строки \texttt{S};
\item \texttt{SetLength(S, N)} --- устанавливает длину строки \texttt{S} равной \texttt{N} символов (с помощью \texttt{SetLength} можно удлинять или укорачивать строки).
\end{enumerate}


\clearpage
\section*{Рубежный контроль №3}
\subsection*{Вариант №2}
\paragraph*{2.1.}
Задана строка, состоящая из слов, разделённых пробелами. Найти количество слов.
\begin{verbatim}
pen  pineapple apple   pen
4
\end{verbatim}
\paragraph*{1.2.}
Задан массив записей (\texttt{record}) с информацией о студентах (всего $ 1 \le N \le 10 $ записей). Каждый студент имеет имя, фамилию и количество набранных за контрольную работу баллов (от 0 до 20).
Вывести информацию о студенте по заданной фамилии (если фамилии повторяются --- вывести всех, --- если нет ни одного, то написать соответствующее сообщение).
\paragraph*{2.3.}
Написать собственную реализацию процедуры \texttt{Delete}. Cравнить результаты работы стандартного и своего вариантов процедуры для произвольной строки.
\begin{verbatim}
procedure MyDelete ( var Source : string; StartChar : Integer; Count : Integer ) 
var
...
begin
...
end;
\end{verbatim}
При написании своей процедуры работать со строкой как с массивом символов (с доступом по индексу, например, \texttt{S[3]}). Также можно использовать стандартные функции:
\begin{enumerate}
\item \texttt{Length(S)} --- возвращает текущую длину строки \texttt{S};
\item \texttt{SetLength(S, N)} --- устанавливает длину строки \texttt{S} равной \texttt{N} символов (с помощью \texttt{SetLength} можно удлинять или укорачивать строки).
\end{enumerate}


\clearpage
\section*{Рубежный контроль №3}
\subsection*{Вариант №3}
\paragraph*{3.1.}
Задан массив из $1 \le N \le 10$ строк. Сформировать новую строку, составленную из элементов массива, разделённых запятыми
\begin{verbatim}
4
apple
pen
pineapple
apple,pen,pineapple
\end{verbatim}
\paragraph*{3.2.}
Задан массив записей (\texttt{record}) с информацией о студентах (всего $ 1 \le N \le 10 $ записей). Каждый студент имеет имя, фамилию и количество набранных за контрольную работу баллов (от 0 до 20).
Отсортировать массив по количеству набранных баллов по убыванию.
\paragraph*{3.3.}
Написать собственную реализацию процедуры \texttt{Insert}. Cравнить результаты работы стандартного и своего вариантов процедуры для произвольной строки.
\begin{verbatim}
procedure MyInsert ( const InsertStr : string; var TargetStr : string; Position : Integer ) ;
var
...
begin
...
end;
\end{verbatim}
При написании своей процедуры работать со строкой как с массивом символов (с доступом по индексу, например, \texttt{S[3]}). Также можно использовать стандартные функции:
\begin{enumerate}
\item \texttt{Length(S)} --- возвращает текущую длину строки \texttt{S};
\item \texttt{SetLength(S, N)} --- устанавливает длину строки \texttt{S} равной \texttt{N} символов (с помощью \texttt{SetLength} можно удлинять или укорачивать строки).
\end{enumerate}

\clearpage
\section*{Рубежный контроль №3}
\subsection*{Вариант №4}
\paragraph*{4.1.}
Задана строка с текстом на английском языке. Вывести самое чаще всего встречающееся в этом тексте слово и число его вхождений. Игнорировать регистр (т.е. "It" и "it" считать за одно и тоже слово).
\begin{verbatim}
The young man suddenly picked up one of Sybil's wet feet, which were drooping over the end of the float, and kissed the arch.
the 4
\end{verbatim}
\paragraph*{4.2.}
Задан массив записей (\texttt{record}) с информацией о студентах (всего $ 1 \le N \le 10 $ записей). Каждый студент имеет имя, фамилию и количество набранных за контрольную работу баллов (от 0 до 20).
Вывести среднее число баллов.
\paragraph*{4.3.}
Написать собственную реализацию функции \texttt{Concat}. Cравнить результаты работы стандартного и своего вариантов функции для произвольной строки.
\begin{verbatim}
function MyConcat ( const String1, String2 : string ) : string;
var
...
begin
...
end;
\end{verbatim}
При написании своей функции работать со строкой как с массивом символов (с доступом по индексу, например, \texttt{S[3]}). Также можно использовать стандартные функции:
\begin{enumerate}
\item \texttt{Length(S)} --- возвращает текущую длину строки \texttt{S};
\item \texttt{SetLength(S, N)} --- устанавливает длину строки \texttt{S} равной \texttt{N} символов (с помощью \texttt{SetLength} можно удлинять или укорачивать строки).
\end{enumerate}



\end{document}