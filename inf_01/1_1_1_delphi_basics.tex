\documentclass[12pt,a4paper]{report}

\usepackage[left=2cm,right=2cm,top=3cm,bottom=3cm,bindingoffset=0cm,pdftex]{geometry}
\usepackage[utf8]{inputenc}
\usepackage{indentfirst}

\usepackage{color}
\usepackage{listings}
\usepackage{amsmath}
\usepackage{textcomp} % Для значка градуса \textdegree
\usepackage{amssymb}  % Для значка треугольника
\usepackage{amsmath}

\usepackage[russian]{babel}

\begin{document}
\parindent=1cm
\pagestyle{empty}

\clearpage

\subsubsection*{Общие требования к программам}
\begin{enumerate}
\item Название программы может состоять только из букв и цифр, должно начинаться с буквы и записывается в виде CamelCase. Имя программы должно отражать решаемую задачу.
Пример: программа вычисления площади круга \texttt{program CircleSquare;}.
\end{enumerate}

\paragraph*{1.1.1.} Задан цилиндр высотой $h$ и радиусом оснований $r$. Вычислить объём цилиндра $V = \pi r^2 h$ и площадь боковой поверхности $S = 2 \pi r h$.

\paragraph*{1.1.2.} Задана сфера с радиусом $r$. Вычислить объём сферы $V = \frac {4} {3} \pi r^3$ и площадь поверхности $S = 4 \pi r^2$.

\paragraph*{1.1.3.} Задано кольцо с внешним радиусом $r1$ и внутренним радиусом $r2$ ($r1 > r2$). Вычислить площадь кольца. (Площадь круга с радиусом $r$ равна $S = \pi r^2$).

\end{document}