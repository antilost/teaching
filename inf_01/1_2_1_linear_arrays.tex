\documentclass[12pt,a4paper]{report}
%\documentclass[12pt,a5paper,landscape]{report}

\usepackage[left=1.5cm,right=1.5cm,top=1.5cm,bottom=1.5cm,bindingoffset=0cm,pdftex]{geometry}
\usepackage[utf8]{inputenc}
\usepackage{indentfirst}

\usepackage{color}
\usepackage{listings}
\usepackage{amsmath}
\usepackage{textcomp} % Для значка градуса \textdegree
\usepackage{amssymb}  % Для значка треугольника
\usepackage{amsmath}

\usepackage[russian]{babel}

\begin{document}
\parindent=1cm
\pagestyle{empty}


%\clearpage
%\paragraph*{Задача №1}
%Создание массива, замена значений в массиве.
%\paragraph*{Задача №2}
%Поиск в массиве.
%\paragraph*{Задача №3}
%Удаление и добавление элементов в массив.


\clearpage
\subsection*{Вариант 1}
\paragraph*{Задача №1}
Задано натуральное число $n <= 20$. Сформируйте массив $A_n$, в котором $A_i = 3 \sin {  \frac {2 \pi (i - 1)} {n} }, i = 1..n$.
\paragraph*{Задача №2}
Заданы натуральные числа $n, m <= 15$ и два массива целых чисел $A_n, B_m$. Выберите массив, в котором больше чисел, кратных заданному целому числу $p > 1$.
\paragraph*{Задача №3}
Задано натуральное число $n <= 20$, массив целых чисел $C_n$ и целые числа $a, b, a < b$. Удалите из массива $C_n$ все элементы, значения которых не входят в интервал $[a; b]$.


\subsection*{Вариант 2}
\paragraph*{Задача №1}
Задано натуральное число $n <= 15$ и массив целых чисел $B_n$. В массиве $B_n$ замените каждый элемент $B_i < 0$ на $\frac {|B_i|} {2}, i = 1..n$.
\paragraph*{Задача №2}
Задано натуральное число $n <= 15$ и массив целых чисел $B_n$. Выведите индексы (порядковые номера) всех локальных минимумов, т.е. таких элементов $B_i$, что $B_{i-1} > B_{i} < B_{i+1}, i=2..n-1$.
\paragraph*{Задача №3}
Задано натуральное число $n <= 20$ и массив целых чисел $C_n$. В массиве $C_n$ найдите наибольшее значение и удалите все элементы, равные этому значению.


\subsection*{Вариант 3}
\paragraph*{Задача №1}
Задано натуральное число $n <= 12$ и целое число $p >= 0$. Сформируйте массив $C_n$ как последовательность $C_n = p, p-1, ..., 2, 1, 0, p, p-1, ..., 2, 1, 0, ...$, состоящую ровно из $n$ элементов.
\paragraph*{Задача №2}
Заданы натуральные числа $n, m <= 15$ и два массива целых чисел $A_n, B_m$. В каком массиве находится наибольшее значение?
\paragraph*{Задача №3}
Задано натуральное число $n <= 20$ и массив целых чисел $C_n$. Удалите из массива $C_n$ все элементы, значения которых кратны заданному целому числу $p > 1$.


\subsection*{Вариант 4}
\paragraph*{Задача №1}
Задано натуральное число $n <= 20$. Сформируйте массив $A_n$, в котором $A_i = 2 \cos { \frac {2 \pi (i - 1)} {n} }, i = 1..n$.
\paragraph*{Задача №2}
Задано натуральное число $n <= 15$ и массив целых чисел $B_n$. Выведите индексы (порядковые номера) всех локальных максимумов, т.е. таких элементов $B_i$, что $B_{i-1} < B_{i} > B_{i+1}, i=2..n-1$.
\paragraph*{Задача №3}
Задано натуральное число $n <= 20$ и массив целых чисел $C_n$. Вычислите среднее арифметическое $m$ элементов массива $C_n$ и удалите все элементы, значения которых не превышают $m$.


\clearpage
\subsection*{Вариант 5}
\paragraph*{Задача №1}
Задано натуральное число $n <= 15$ и массив целых чисел $B_n$. Каждый элемент массива замените остатком от деления значения этого элемента на заданное целое число $p > 1$.
\paragraph*{Задача №2}
Заданы натуральные числа $n, m <= 15$ и два массива целых чисел $A_n, B_m$. В каком массиве среднее арифметическое значений элементов меньше?
\paragraph*{Задача №3}
Задано натуральное число $n <= 20$ и массив целых чисел $C_n$. Удалите из массива $C_n$ все элементы, индексы (порядковые номера) которых кратны заданному числу $p, 1 < p <= n$ (удалите каждый $p\text{-й}$ элемент).


\subsection*{Вариант 6}
\paragraph*{Задача №1}
Задано натуральное число $n <= 12$ и действительное число $p$. Сформируйте массив $C_n$ как последовательность $C_n = p, \frac{p}{2}, ..., \frac{p}{n-1}, \frac{p}{n}$.
\paragraph*{Задача №2}
Заданы натуральные число $n, p, q, \text{ причём } n <= 18, 1 <= p < q <= n$ и массив целых чисел $D_n$. Найдите наибольшее значение среди элементов $D_p, D_{p+1}, ..., D_{q-1}, D_q$.
\paragraph*{Задача №3}
Задано натуральное число $n <= 20$ и массив целых чисел $C_n$. Удалите из массива $C_n$ все неотрицательные элементы.


\subsection*{Вариант 7}
\paragraph*{Задача №1}
Задано натуральное число $n <= 20$. Сформируйте массив $A_n$, в котором $A_i = \cos \alpha \sin \alpha, \alpha = \frac {2\pi i} {n}, i=1..n$.
\paragraph*{Задача №2}
Заданы натуральные числа $n, m <= 15$ и два массива целых чисел $A_n, B_m$. В каком массиве находится наименьшее значение?
\paragraph*{Задача №3}
Задано натуральное число $n <= 20$ и массив целых чисел $C_n$. Удалите из массива $C_n$ все элементы, значения которых не кратны заданному целому числу $p > 1$.


\subsection*{Вариант 8}
\paragraph*{Задача №1}
Задано натуральное число $n <= 12$ и целое чётное число $p > 1$. Сформируйте массив $D_n$ как последовательность $D_n = 0, 2, ..., p-2, p, 0, 2, ..., p-2, p$, состоящую ровно из $n$ элементов.
\paragraph*{Задача №2}
Задано натуральное число $n <= 15$ и массив целых чисел $B_n$. Выведите индексы (порядковые номера) наибольшего и наименьшего элементов массива.
\paragraph*{Задача №3}
Задано натуральное число $n <= 20$ и массив целых чисел $C_n$. Удалите из массива $C_n$ все положительные элементы.


\clearpage
\subsection*{Вариант 9}
\paragraph*{Задача №1}
Задано натуральное чётное число $n <= 16$ и массив целых чисел $B_n$. Обеспечьте выполнение равенства $B_{2j-1} <= B_{2j}$ для каждой пары элементов $B_{2j-1}, B_{2j}, j=1..\frac{n}{2}$, при необходимости поменяв значения элементов между собой.
\paragraph*{Задача №2}
Заданы натуральные числа $n, m <= 15$ и два массива целых чисел $A_n, B_m$, в каждом из которых элементы упорядочены по возврастанию (пользователь изначально вводит неубывающую последовательность). Сформируйте третий массив $C_{n+m}$ из элементов $A_n$ и $B_m$, в котором элементы также будут упорядочены по возрастанию.
\paragraph*{Задача №3}
Задано натуральное число $n <= 20$ и массив целых чисел $C_n$. Переставьте в конец массива $C_n$ все нулевые элементы $C_i = 0$, не меняя порядок остальных элементов.


\end{document}