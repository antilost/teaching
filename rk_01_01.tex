\documentclass[12pt,a5paper,landscape]{article}

\usepackage[left=1cm,right=1cm,top=1cm,bottom=1cm,bindingoffset=0cm,pdftex]{geometry}

\usepackage{color}
\usepackage{listings}

\usepackage[utf8]{inputenc}
\usepackage[english,russian]{babel}
\usepackage[T2A]{fontenc}

\renewcommand{\lstlistingname}{Листинг}

\lstset{
  literate={а}{{\selectfont\char224}}1
           {б}{{\selectfont\char225}}1
           {в}{{\selectfont\char226}}1
           {г}{{\selectfont\char227}}1
           {д}{{\selectfont\char228}}1
           {е}{{\selectfont\char229}}1
           {ё}{{\"e}}1
           {ж}{{\selectfont\char230}}1
           {з}{{\selectfont\char231}}1
           {и}{{\selectfont\char232}}1
           {й}{{\selectfont\char233}}1
           {к}{{\selectfont\char234}}1
           {л}{{\selectfont\char235}}1
           {м}{{\selectfont\char236}}1
           {н}{{\selectfont\char237}}1
           {о}{{\selectfont\char238}}1
           {п}{{\selectfont\char239}}1
           {р}{{\selectfont\char240}}1
           {с}{{\selectfont\char241}}1
           {т}{{\selectfont\char242}}1
           {у}{{\selectfont\char243}}1
           {ф}{{\selectfont\char244}}1
           {х}{{\selectfont\char245}}1
           {ц}{{\selectfont\char246}}1
           {ч}{{\selectfont\char247}}1
           {ш}{{\selectfont\char248}}1
           {щ}{{\selectfont\char249}}1
           {ъ}{{\selectfont\char250}}1
           {ы}{{\selectfont\char251}}1
           {ь}{{\selectfont\char252}}1
           {э}{{\selectfont\char253}}1
           {ю}{{\selectfont\char254}}1
           {я}{{\selectfont\char255}}1
           {А}{{\selectfont\char192}}1
           {Б}{{\selectfont\char193}}1
           {В}{{\selectfont\char194}}1
           {Г}{{\selectfont\char195}}1
           {Д}{{\selectfont\char196}}1
           {Е}{{\selectfont\char197}}1
           {Ё}{{\"E}}1
           {Ж}{{\selectfont\char198}}1
           {З}{{\selectfont\char199}}1
           {И}{{\selectfont\char200}}1
           {Й}{{\selectfont\char201}}1
           {К}{{\selectfont\char202}}1
           {Л}{{\selectfont\char203}}1
           {М}{{\selectfont\char204}}1
           {Н}{{\selectfont\char205}}1
           {О}{{\selectfont\char206}}1
           {П}{{\selectfont\char207}}1
           {Р}{{\selectfont\char208}}1
           {С}{{\selectfont\char209}}1
           {Т}{{\selectfont\char210}}1
           {У}{{\selectfont\char211}}1
           {Ф}{{\selectfont\char212}}1
           {Х}{{\selectfont\char213}}1
           {Ц}{{\selectfont\char214}}1
           {Ч}{{\selectfont\char215}}1
           {Ш}{{\selectfont\char216}}1
           {Щ}{{\selectfont\char217}}1
           {Ъ}{{\selectfont\char218}}1
           {Ы}{{\selectfont\char219}}1
           {Ь}{{\selectfont\char220}}1
           {Э}{{\selectfont\char221}}1
           {Ю}{{\selectfont\char222}}1
           {Я}{{\selectfont\char223}}1
}
\usepackage{courier}

\usepackage{indentfirst}
\usepackage{amsmath}
\usepackage{textcomp} % Для значка градуса \textdegree
\usepackage{amssymb}  % Для значка треугольника
\usepackage{amsmath}


\begin{document}
\parindent=1cm
\pagestyle{empty}

\lstset{ %
%language=Delphi,               % выбор языка для подсветки (здесь это С)
basicstyle=\small\ttfamily,   % размер и начертание шрифта для подсветки кода
numbers=left,                  % где поставить нумерацию строк (слева\справа)
numberstyle=\tiny,             % размер шрифта для номеров строк
stepnumber=1,                  % размер шага между двумя номерами строк
numbersep=5pt,                 % как далеко отстоят номера строк от подсвечиваемого кода
backgroundcolor=\color{white}, % цвет фона подсветки - используем \usepackage{color}
showspaces=false,              % показывать или нет пробелы специальными отступами
showstringspaces=false,        % показывать или нет пробелы в строках
showtabs=false,                % показывать или нет табуляцию в строках
frame=single,                  % рисовать рамку вокруг кода
tabsize=2,                     % размер табуляции по умолчанию равен 2 пробелам
captionpos=t,                  % позиция заголовка вверху [t] или внизу [b] 
breaklines=true,               % автоматически переносить строки (да\нет)
breakatwhitespace=false,       % переносить строки только если есть пробел
escapeinside={\%*}{*)}         % если нужно добавить комментарии в коде
}


\clearpage
\section*{Рубежный контроль №1}
\subsection*{Вариант №1}
\subsubsection*{Задача №1}
Дано вещественное число $a$ -- длина ребра куба. Вычислить площадь полной поверхности $S$ и объём куба $V$.
\subsubsection*{Задача №2}
Последовательность чисел Фибоначчи определяется так: $F_0 = 0, F_1 = 1, F_i = F_{i-1} + F_{i-2}$ при $i \ge 2$. Дано целое число $n \ge 0$, вычислить $F_n$.


\clearpage
\section*{Рубежный контроль №1}
\subsection*{Вариант №2}
\subsubsection*{Задача №1}
Заданы вещественные числа $a, b, \Delta x$. Вывести таблицу значений функций $f_1(x), f_2(x), f_3(x)$ для $x$, изменяющегося в интервале $[a;b]$ с шагом $\Delta x$.
\begin{itemize}
\item $f_1(x) = \sin{x}$;
\item $f_2(x) = 2 \log_2{|x + 1|}$;
\item $f_3(x) = \frac{\sin^2{x}}{3} - \frac{\cos^2{(x+\frac{\pi}{2})}}{4}$.
\end{itemize}
\subsubsection*{Задача №2}
Дано натуральное число $n$ и $n$ вещественных чисел; числа вводятся по одному на строку. После ввода каждого нового числа вывести среднее арифметическое всех ранее введённых чисел.


\clearpage
\section*{Рубежный контроль №1}
\subsection*{Вариант №3}
\subsubsection*{Задача №1}
Найти наименьшее среди трёх целых чисел $p$, $q$ и $r$.
\subsubsection*{Задача №2}
Заданы целые числа $a, b$ и $d$, причём $a < b, d \ge 1$. Вывести квадраты всех целых чисел в интервале от $a$ до $b$ с шагом $d$, т.е. последовательность чисел $a^2, (a+d)^2, \ldots, (a + id)^2, \ldots$, где последнее значение не превышает $b$.


\clearpage
\section*{Рубежный контроль №1}
\subsection*{Вариант №4}
\subsubsection*{Задача №1}
Заданы вещественные числа $a, b, \Delta x$. Вывести таблицу значений функций $f_1(x), f_2(x), f_3(x)$ для $x$, изменяющегося в интервале $[a;b]$ с шагом $\Delta x$.
\begin{itemize}
\item $f_1(x) = \cos{x}$;
\item $f_2(x) = 0.1e^{|x|}$;
\item $f_3(x) = \frac{\cos^3{(x+\frac{\pi}{2})}}{5} - \frac{\sin^3{x}}{7} $.
\end{itemize}
\subsubsection*{Задача №2}
Дано натуральное число $n$. Вычислить $(n + \frac{1}{1^2})(n-1 + \frac{1}{2^2}) \ldots (1 + \frac{1}{n^2})$.


\clearpage
\section*{Рубежный контроль №1}
\subsection*{Вариант №5}
\subsubsection*{Задача №1}
Заданы 8 вещественных чисел -- координаты точек $A, B, C$ и $O$: $x_a, y_a, x_b, y_b, x_c, y_c, x_o, y_o$. Выяснить, какая из точек $A$, $B$ или $C$ находится ближе всего к точке $O$.
\subsubsection*{Задача №2}
Последовательность чисел Фибоначчи определяется так: $F_0 = 0, F_1 = 1, F_i = F_{i-1} + F_{i-2}$ при $i \ge 2$. Дано целое число $n \ge 0$, вычислить $F_n$.


\clearpage
\section*{Рубежный контроль №1}
\subsection*{Вариант №6}
\subsubsection*{Задача №1}
Заданы вещественные числа $a, b, \Delta x$. Вывести таблицу значений функций $f_1(x), f_2(x), f_3(x)$ для $x$, изменяющегося в интервале $[a;b]$ с шагом $\Delta x$.
\begin{itemize}
\item $f_1(x) = x^3$;
\item $f_2(x) = 0.25e^{x}$;
\item $f_3(x) = 4x^5 - 3x^3 +7x^2+2x-3$.
\end{itemize}
\subsubsection*{Задача №2}
Дано натуральное число $n$ и $n$ вещественных чисел; числа вводятся по одному на строку. После ввода каждого нового числа вывести среднее арифметическое всех ранее введённых чисел.


\clearpage
\section*{Рубежный контроль №1}
\subsection*{Вариант №7}
\subsubsection*{Задача №1}
Найти наибольшее среди трёх целых чисел $p$, $q$ и $r$.
\subsubsection*{Задача №2}
Заданы целые числа $a, b$ и $d$, причём $a < b, d \ge 1$. Вывести квадраты всех целых чисел в интервале от $a$ до $b$ с шагом $d$, т.е. последовательность чисел $a^2, (a+d)^2, \ldots, (a + id)^2, \ldots$, где последнее значение не превышает $b$.


\clearpage
\section*{Рубежный контроль №1}
\subsection*{Вариант №8}
\subsubsection*{Задача №1}
Заданы вещественные числа $a, b, \Delta x$. Вывести таблицу значений функций $f_1(x), f_2(x), f_3(x)$ для $x$, изменяющегося в интервале $[a;b]$ с шагом $\Delta x$.
\begin{itemize}
\item $f_1(x) = -x^2$;
\item $f_2(x) = \frac {\ln |x|} {12} $;
\item $f_3(x) = \sqrt { (|x-1|)^3(x+4)^2(|x|+3) }$.
\end{itemize}
\subsubsection*{Задача №2}
Дано натуральное число $n$. Вычислить $(n + \frac{1}{1^2})(n-1 + \frac{1}{2^2}) \ldots (1 + \frac{1}{n^2})$.



\end{document}
