\documentclass[12pt,a5paper,landscape]{article}

\usepackage[left=1.2cm,right=1.2cm,top=1.2cm,bottom=2cm,bindingoffset=0cm,pdftex]{geometry}

\usepackage{color}
\usepackage{listings}

\usepackage[utf8]{inputenc}
\usepackage[english,russian]{babel}
\usepackage[T2A]{fontenc}

\renewcommand{\lstlistingname}{Листинг}

\lstset{
  literate={а}{{\selectfont\char224}}1
           {б}{{\selectfont\char225}}1
           {в}{{\selectfont\char226}}1
           {г}{{\selectfont\char227}}1
           {д}{{\selectfont\char228}}1
           {е}{{\selectfont\char229}}1
           {ё}{{\"e}}1
           {ж}{{\selectfont\char230}}1
           {з}{{\selectfont\char231}}1
           {и}{{\selectfont\char232}}1
           {й}{{\selectfont\char233}}1
           {к}{{\selectfont\char234}}1
           {л}{{\selectfont\char235}}1
           {м}{{\selectfont\char236}}1
           {н}{{\selectfont\char237}}1
           {о}{{\selectfont\char238}}1
           {п}{{\selectfont\char239}}1
           {р}{{\selectfont\char240}}1
           {с}{{\selectfont\char241}}1
           {т}{{\selectfont\char242}}1
           {у}{{\selectfont\char243}}1
           {ф}{{\selectfont\char244}}1
           {х}{{\selectfont\char245}}1
           {ц}{{\selectfont\char246}}1
           {ч}{{\selectfont\char247}}1
           {ш}{{\selectfont\char248}}1
           {щ}{{\selectfont\char249}}1
           {ъ}{{\selectfont\char250}}1
           {ы}{{\selectfont\char251}}1
           {ь}{{\selectfont\char252}}1
           {э}{{\selectfont\char253}}1
           {ю}{{\selectfont\char254}}1
           {я}{{\selectfont\char255}}1
           {А}{{\selectfont\char192}}1
           {Б}{{\selectfont\char193}}1
           {В}{{\selectfont\char194}}1
           {Г}{{\selectfont\char195}}1
           {Д}{{\selectfont\char196}}1
           {Е}{{\selectfont\char197}}1
           {Ё}{{\"E}}1
           {Ж}{{\selectfont\char198}}1
           {З}{{\selectfont\char199}}1
           {И}{{\selectfont\char200}}1
           {Й}{{\selectfont\char201}}1
           {К}{{\selectfont\char202}}1
           {Л}{{\selectfont\char203}}1
           {М}{{\selectfont\char204}}1
           {Н}{{\selectfont\char205}}1
           {О}{{\selectfont\char206}}1
           {П}{{\selectfont\char207}}1
           {Р}{{\selectfont\char208}}1
           {С}{{\selectfont\char209}}1
           {Т}{{\selectfont\char210}}1
           {У}{{\selectfont\char211}}1
           {Ф}{{\selectfont\char212}}1
           {Х}{{\selectfont\char213}}1
           {Ц}{{\selectfont\char214}}1
           {Ч}{{\selectfont\char215}}1
           {Ш}{{\selectfont\char216}}1
           {Щ}{{\selectfont\char217}}1
           {Ъ}{{\selectfont\char218}}1
           {Ы}{{\selectfont\char219}}1
           {Ь}{{\selectfont\char220}}1
           {Э}{{\selectfont\char221}}1
           {Ю}{{\selectfont\char222}}1
           {Я}{{\selectfont\char223}}1
}
\usepackage{courier}

\usepackage{indentfirst}
\usepackage{amsmath}
\usepackage{textcomp} % Для значка градуса \textdegree
\usepackage{amssymb}  % Для значка треугольника
\usepackage{amsmath}


\begin{document}
\parindent=1cm
\pagestyle{empty}

\lstset{ %
%language=Delphi,               % выбор языка для подсветки (здесь это С)
basicstyle=\small\ttfamily,   % размер и начертание шрифта для подсветки кода
%numbers=left,                  % где поставить нумерацию строк (слева\справа)
numberstyle=\tiny,             % размер шрифта для номеров строк
stepnumber=1,                  % размер шага между двумя номерами строк
numbersep=5pt,                 % как далеко отстоят номера строк от подсвечиваемого кода
backgroundcolor=\color{white}, % цвет фона подсветки - используем \usepackage{color}
showspaces=false,              % показывать или нет пробелы специальными отступами
showstringspaces=false,        % показывать или нет пробелы в строках
showtabs=false,                % показывать или нет табуляцию в строках
frame=single,                  % рисовать рамку вокруг кода
tabsize=2,                     % размер табуляции по умолчанию равен 2 пробелам
%captionpos=t,                  % позиция заголовка вверху [t] или внизу [b] 
breaklines=true,               % автоматически переносить строки (да\нет)
breakatwhitespace=false,       % переносить строки только если есть пробел
escapeinside={\%*}{*)}         % если нужно добавить комментарии в коде
}


\clearpage
\section*{Рубежный контроль №2}
\subsection*{Демонстрационный вариант №1}
\subsubsection*{Задача №1}
Дано целое число $n$, $1 \le n \le 10$, квадратная матрица вещественных чисел $M_{n \times n}$, и вещественные числа $a, b, a < b$.
Вычислить сумму элементов главной диагонали матрицы $M$, значения которых входят в интервал $[a; b]$.
\subsubsection*{Задача №2}
Даны два целых числа $n, m$, $1 \le n \le 10, 1 \le m \le 10$ и два массива целых чисел $A_{n}, B_{m}$ (длина массива $A$ равна $n$, длина массива $B$ равна $m$). Значения в массивах $A$, $B$ вводятся изначально упорядоченными по возрастанию.
Получить новый массив целых чисел $C_{n+m}$, который состоит из элементов массивов $A$, $B$, также упорядоченных по возрастанию.
Не производить сортировку массивов, новый массив получить за один проход (цикл) по каждому из заданных массивов.

\begin{lstlisting}[caption={}]
A: 14 35 48 77 81
B: 16 22 35 40 48 93
C: 14 16 22 35 35 40 48 48 77 81 93
\end{lstlisting}

\clearpage
\section*{Рубежный контроль №2}
\subsection*{Демонстрационный вариант №2}
\subsubsection*{Задача №1}
Даное целое число $n$, $1 \le n \le 15$ и массив целых чисел $B_{n}$ длиной $n$. За один проход (цикл) по массиву $B$ удалить все нечётные элементы.
\subsubsection*{Задача №2}
Даны целые числа $n, m, 1 \le n \le 15, 1 \le m \le 20$, определяющие размерность матрицы целых чисел $M_{n \times m}$ ($n$ строк, $m$ столбцов). Получить матрицу $M$, заполненную квадратами первых $nm$ натуральных чисел по строкам змейкой, начиная с верхнего левого угла.
\begin{lstlisting}[caption={}]
n = 4 m = 5
  1   4   9  16  25
100  81  64  49  36
121 144 169 196 225
400 361 324 289 256
\end{lstlisting}

\end{document}
