\section{Семинар №1}

\subsection{Список рекомендованной литературы}
\begin{enumerate}
\item Фленов М.Е. Библия Delphi --- СПб.: БХВ-Петербург, 2011. --- 688 с.;
\item Алексеев Ю.Е., Ваулин А.С., Куров А.В. Практикум по программированию: обработка числовых данных: Учеб. пособие / Под ред. Б.Г. Трусова. --- М.: Изд-во МГТУ им. Н.Э. Баумана, 2008. --- 288 с.;
\item Д.М. Ушаков, Т.А. Юркова Паскаль для школьников --- СПб.: Питер, 2011. --- 320 с.
\end{enumerate}

\subsection{Создание консольных приложений в среде разработки Lazarus}
\subsubsection{Проект}
https://lazarus-rus.ru
\subsubsection{Программа}
\subsubsection{Компиляция}

\subsection{Структура программы}

\subsubsection{Ключевые слова}
\begin{itemize}
\item \texttt{program}
\item \texttt{begin}
\item \texttt{end}
\end{itemize}

\subsubsection{Комментарии}
\begin{itemize}
\item Многострочный комментарий
\begin{verbatim}
{
  Все символы между фигурными скобками -- комментарий
}
\end{verbatim}
\item Однострочный комментарий
\begin{verbatim}
// Всё после двух символов косой черты (слешей) и до конца строки -- комментарий
\end{verbatim}
\end{itemize}

\subsubsection{Операторы ввода и вывода}
\begin{itemize}
\item \texttt{Write}
\item \texttt{WriteLn}
\item \texttt{Read}
\item \texttt{ReadLn}
\end{itemize}

\subsubsection{Пример программы с выводом на экран}
Задача: вывести на экран строку "Hello, world!".
\begin{lstlisting}[label=some-code,caption={Текст программы HelloWorld}]
program HelloWorld; // Заголовок

{
  Простое консольное приложение
}

begin
  WriteLn('Hello, world!'); // Выводим сообщение пользователю
  ReadLn; // А эта строка нужна, чтобы окно программы сразу не закрылось
end. // Конец текста программы
\end{lstlisting}


\subsubsection{Пример программы с вычислениями}
Даны два вещественных (действительных) числа $a$ и $b$. Получить их сумму, разность и произведение
\begin{lstlisting}[label=some-code,caption={Текст программы MathOperations}]
program MathOperations;

var // блок определения переменных
  A, B: Real;
  S: Real; // сумма
  D: Real; // разность
  P: Real; // произведение
begin // начало основной программы
  Write('Enter A: '); // вывод текста 'Enter A' на экран без переноса строки
  ReadLn(A); // ввод значения переменной A

  Write('Enter B: '); // вывод текста 'Enter B' на экран без переноса строки
  ReadLn(B); // ввод значения переменной B

  S := A + B; // вычисление суммы, результат сохраняется в переменную S
  D := A - B; // результат A - B записывается в D
  P := A * B; // result A * B will be written at P

  WriteLn('Sum is ', S);
  WriteLn('Difference is ', D);
  WriteLn('Product is ', P);

  ReadLn;
end. // конец текста программы
\end{lstlisting}

\subsubsection{Переменные}
\subsubsection{Константы}






% Основные стандартные типы данных

% Оператор присваивания

% Выражения

% Процедуры ввода - вывода

% Разработка программ линейной структуры.


\subsection*{Общие требования к программам}
\begin{enumerate}
\item Название программы может состоять только из букв и цифр, должно начинаться с буквы и записывается в виде CamelCase. Имя программы должно отражать решаемую задачу.
Пример: программа вычисления площади круга \texttt{program CircleSquare;}.
\end{enumerate}


% Знакомство с системой программирования Delphi (консольный режим). Структура программы. Основные стандартные типы данных. Оператор присваивания. Выражения. Процедуры ввода - вывода. Разработка программ линейной структуры.
% Операторы: условный, составной и выбора. Логические операции. Разработка программ разветвляющейся структуры.
% Операторы цикла (с известным и неизвестным числом повторений). Разработка программ циклической структуры.




\paragraph*{1.1.1.} Задан цилиндр высотой $h$ и радиусом оснований $r$. Вычислить объём цилиндра $V = \pi r^2 h$ и площадь боковой поверхности $S = 2 \pi r h$.

\paragraph*{1.1.2.} Задана сфера с радиусом $r$. Вычислить объём сферы $V = \frac {4} {3} \pi r^3$ и площадь поверхности $S = 4 \pi r^2$.

\paragraph*{1.1.3.} Задано кольцо с внешним радиусом $r1$ и внутренним радиусом $r2$ ($r1 > r2$). Вычислить площадь кольца. (Площадь круга с радиусом $r$ равна $S = \pi r^2$).
