\documentclass[12pt,a4paper]{report}

\usepackage[left=2cm,right=2cm,top=3cm,bottom=3cm,bindingoffset=0cm,pdftex]{geometry}
\usepackage[utf8]{inputenc}
\usepackage[russian]{babel}
\usepackage{indentfirst}

\usepackage{color}
\usepackage{listings}
\usepackage{amsmath}
%\usepackage{textcomp} % Для значка градуса
\usepackage{amssymb}  % Для значка треугольника
\usepackage{amsmath}

\begin{document}
\parindent=1cm
\pagestyle{empty}

\lstset{ language=Pascal, basicstyle=\small\ttfamily, numbers=left, numberstyle=\tiny, stepnumber=1, numbersep=5pt, extendedchars=\true, showstringspaces=false, breakatwhitespace=true, frame=single, keepspaces=true }
\clearpage
theory

\clearpage
\clearpage
\subsubsection*{ДЗ №2.1}
\paragraph*{ДЗ 2.1.1} Дано натуральное число $n \le 20$ и последовательность целых чисел $a_1, ..., a_n$. Если знаки чисел последовательности чередуются ($+,-,+,-, ...$ или $-,+,-,+, ....$), то найти количество отрицательных чисел; иначе, вычислить произведение всех чисел последовательности, не включая нули.
\paragraph*{ДЗ 2.1.2} Дано натуральное число $n \le 20$ и последовательность целых чисел $a_1, ..., a_n$. Если все числа, стоящие на чётных позициях $a_2, a_4, a_6, ...$, --- чётные, то найти сумму всех чисел последовательности; иначе, вычислить минимум среди чисел, стоящих на нечётных позициях $min(a_1, a_3, ...)$.
\paragraph*{ДЗ 2.1.3} Дано натуральное число $n \le 20$ и последовательность целых чисел $a_1, ..., a_n$. Если $a_1 > a_2 > ... > a_n$, то вывести среднее геометрическое чисел последовательности; иначе, найти минимум $min(a_1, ..., a_n)$.
\paragraph*{ДЗ 2.1.4} Дано натуральное число $n \le 20$ и последовательность целых чисел $a_1, ..., a_n$. Если $a_1 \le a_2 \le ... \le a_n$, то вывести среднее арифметическое чисел последовательности; иначе, найти максимум $max(a_1, ..., a_n)$.
\paragraph*{ДЗ 2.1.5} Дано натуральное число $n \le 20$ и последовательность целых чисел $a_1, ..., a_n$. Найти количество нечётных чисел в последовательности. Условный оператор if .. then не использовать.
\paragraph*{ДЗ 2.1.6} Дано натуральное число $n \le 20$ и последовательность целых чисел $a_1, ..., a_n$. Удалить из последовательности все числа $a_i$ со значением $max(a_1, ..., a_n)$. Решить задачу, используя не более двух проходов по последовательности $a_1, ..., a_n$.
\paragraph*{ДЗ 2.1.7} Даны натуральные числа $n \le 20, m \le n$ и последовательность целых чисел $a_1, ..., a_n$. Требуется домножить все члены подпоследовательности $a_1, a_2, ..., a_{m}$ на квадрат её наименьшего члена, если $a_1 \ge 0$, и на квадрат её наибольшего члена, если $a_1 < 0$.
\paragraph*{ДЗ 2.1.8} Даны натуральные числа $n \le 20, m \le n$ и две последовательности целых чисел $a_1, ..., a_n$, $b_1, ..., b_m$. Проверить, что члены последовательности $b_1, ..., b_m$ входят в последовательность $a_1, ..., a_n$, с сохранением порядка, но не обязательно непрерывно. Например, $b=2,4,8$ является подходящей последовательностью для $a=1,2,3,4,5,6,7,8,9,10$.
\paragraph*{ДЗ 2.1.9} Дано натуральное число $n \le 20$ и последовательность действительных чисел $a_1, ..., a_n$. Получить последовательность $b_1, b_2, ..., b_n$, где $b_i$ --- среднее арифметическое всех членов последовательности $a_1, ..., a_{i-1}, ..., a_{i+1}, ..., a_n$ (без $a_i$). Решить задачу, используя не более двух проходов по последовательности $a_1, ..., a_n$.
\paragraph*{ДЗ 2.1.10} Дано натуральное число $n \le 20$ и последовательность целых чисел $a_1, ..., a_n$. Найти два числа: которое встречается чаще всех остальных и которое, наоборот, встречается реже всех остальных. Для обоих чисел вывести число повторений. 

\clearpage
\subsubsection*{ДЗ №2.2}
\paragraph*{ДЗ 2.2.1}
\paragraph*{ДЗ 2.2.2}
\paragraph*{ДЗ 2.2.3}
\paragraph*{ДЗ 2.2.4}
\paragraph*{ДЗ 2.2.5}
\paragraph*{ДЗ 2.2.6}
\paragraph*{ДЗ 2.2.7}
\paragraph*{ДЗ 2.2.8}
\paragraph*{ДЗ 2.2.9}
\paragraph*{ДЗ 2.2.X}

\clearpage
\subsubsection*{ЛР №2.1}
% получить чиселки
\paragraph*{ЛР 2.1.1.1} Дано натуральное число $n \le 20$ и последовательность целых чисел $a_1, ..., a_n$. Вычислить среднее арифметическое чисел последовательности и количество чисел, которые превышают его.
\paragraph*{ЛР 2.1.1.2} Дано натуральное число $n \le 20$, последовательность целых чисел $x_1, ..., x_n$ и последовательность действительных чисел $b_1, ..., b_n$. Вычислить $\frac { x_1 b_1 + x_2 b_2 + ... + x_n b_n} { x_1 + x_2 + ... + x_n }$ и найти минимум $min(x_1 b_1, x_2 b_3, ..., x_{n-1} b_{n-1})$.
\paragraph*{ЛР 2.1.1.3} Дано натуральное число $n \le 20$ и последовательность целых чисел $a_1, ..., a_n$. Вычислить минимум $min(a_1, ..., a_n)$, максимум $max(a_1, ..., a_n)$, среднее арифметическое и среднее геометрическое чисел последовательности.
\paragraph*{ЛР 2.1.1.4} Дано натуральное число $n \le 20$ и две последовательности целых чисел $a_1, ..., a_n$ и $b_1, ..., b_n$. Вычислить $(a_{n-1} + b_2)(a_{n-3} + b_4) ... (a_1 + b_{n})$ и найти максимум $max(a_1, ..., a_n)$. 
\paragraph*{ЛР 2.1.1.5} Дано натуральное число $n \le 20$ и две последовательности целых чисел $a_1, ..., a_n$ и $b_1, ..., b_n$. Вычислить $(a_1 + b_n)(a_2 + b_{n-1}) ... (a_n + b_1)$ и найти минимум $min(b_1, ..., b_n)$.

% получить новую последовательность
\paragraph*{ЛР 2.1.2.1} Дана последовательность действительных чисел $a_1, ..., a_{20}$. \\
Получить новую последовательность $a_{20}, a_{11}, a_{19}, a_{10}, ..., a_{10}, a_1$.
\paragraph*{ЛР 2.1.2.2} Дана последовательность действительных чисел $a_1, ..., a_{20}$. \\
Получить новую последовательность  $a_1, a_3, ..., a_{19}, a_2, a_4, ..., a_{20}$.
\paragraph*{ЛР 2.1.2.3} Дана последовательность действительных чисел $a_1, ..., a_{20}$. \\
Получить новую последовательность  $a_1, a_{11}, a_3, a_{13}, ..., a_9, a_{19}$.
\paragraph*{ЛР 2.1.2.4} Дана последовательность действительных чисел $a_1, ..., a_{20}$. \\
Получить новую последовательность  $a_{12}, a_2, a_{14}, a_4, ..., a_{20}, a_{10}$.
\paragraph*{ЛР 2.1.2.5} Дана последовательность действительных чисел $a_1, ..., a_{20}$. \\
Получить новую последовательность  $a_1, a_{20}, a_3, a_{18}, ..., a_{19}, a_2$.

\clearpage
% удалить/вставить в одном и том же массиве
\paragraph*{ЛР 2.1.3.1} Дано натуральное число $n \le 20$ и последовательность целых чисел $x_1, ..., x_n$. Удалить из последовательности все нули $0$.
\paragraph*{ЛР 2.1.3.2} Даны натуральные числа $n \le 20, p \le 100$ и последовательность целых чисел $x_1, ..., x_n$. После каждого  числа последовательности $x_i$, кратного $p$, вставить произведение $x_i p$.
\paragraph*{ЛР 2.1.3.3} Даны натуральные числа $n \le 20, p \le 100$ и последовательность целых чисел $x_1, ..., x_n$. Удалить из последовательности все числа $x_i$, кратные $p$.
\paragraph*{ЛР 2.1.3.4} Дано натуральное число $n \le 20$ и последовательность целых чисел $x_1, ..., x_n$. После каждого отрицательного числа $x_i<0$ вставить его модуль $|x_i|$.
\paragraph*{ЛР 2.1.3.5} Дано натуральное число $n \le 20$ и последовательность целых чисел $x_1, ..., x_n$. Оставить в последовательности только неотрицательные числа $x_i \ge 0$, а все остальные удалить.


\clearpage
\subsection*{ЛР №2.2}
\noindent % чтобы minipage не был с красной строки
\begin{minipage}{0.75\textwidth}
\paragraph*{ЛР 2.2.1.1} 
Дано натуральное число $n \le 10$. Получить матрицу $A_{n \times n}$ из последовательно идущих по строкам слева направо и сверху вниз первых $n^2$ чисел Фибоначчи. Числа Фибоначчи определяются по формуле:
$F_i=\begin{cases}
1, & \text{при } i \le 2; \\
F_i = F_{i-2} + F_{i-1}, & \text{при } i > 2. \\
\end{cases}$
\end{minipage}
\hfill
\begin{minipage}{0.2\textwidth}
\begin{verbatim}
N = 4
1 1 2 3
5 8 13 21
34 55 89 144
233 377 610 987
\end{verbatim}
\end{minipage}

\subsubsection*{} \noindent
\begin{minipage}{0.75\textwidth}
\paragraph*{ЛР 2.2.1.2}
Даны натуральные числа $m,n \le 10$. Получить матрицу $A_{m \times n}$ из чередующихся нулей и единиц, причём если $m>n$, то элемент первой строки первого столбца будет 0, иначе 1.
\end{minipage}
\hfill
\begin{minipage}{0.2\textwidth}
\begin{verbatim}
M = 2 N = 7
1 0 1 0 1 0 1
0 1 0 1 0 1 0
\end{verbatim}
\end{minipage}

\subsubsection*{} \noindent
\begin{minipage}{0.75\textwidth}
\paragraph*{ЛР 2.2.1.3} 
Даны натуральные числа $m,n \le 10$. Получить матрицу $A_{m \times n}$ из последовательно идущих по столбцам сверху вниз и слева направо первых $mn$ нечётных натуральных чисел.
\end{minipage}
\hfill
\begin{minipage}{0.2\textwidth}
\begin{verbatim}
M = 3 N = 4
1 7 13 19
3 9 15 21
5 11 17 23
\end{verbatim}
\end{minipage}

\subsubsection*{} \noindent
\begin{minipage}{0.75\textwidth}
\paragraph*{ЛР 2.2.1.4} 
Даны натуральные числа $m,n \le 10$. Получить матрицу $A_{m \times n}$ из последовательно идущих по столбцам снизу вверх и слева направо первых $mn$ чётных натуральных чисел.
\end{minipage}
\hfill
\begin{minipage}{0.2\textwidth}
\begin{verbatim}
M = 3 N = 4
6 12 18 24
4 10 16 22
2 8 14 20
\end{verbatim}
\end{minipage}

\subsubsection*{} \noindent
\begin{minipage}{0.75\textwidth}
\paragraph*{ЛР 2.2.1.5} 
Дано натуральное число $n \le 10$. Получить верхнюю треугольную матрицу $A_{n \times n}$, у которой каждый элемент, расположенный ниже главной диагонали, равен нулю, а все остальные -- единицы.
\end{minipage}
\hfill
\begin{minipage}{0.2\textwidth}
\begin{verbatim}
N = 3
1 1 1
0 1 1
0 0 1
\end{verbatim}
\end{minipage}

\subsubsection*{} \noindent
\begin{minipage}{0.75\textwidth}
\paragraph*{ЛР 2.2.1.6} 
Дано натуральное число $n \le 10$. Получить нижнюю треугольную матрицу $A_{n \times n}$, у которой каждый элемент, расположенный выше главной диагонали, равен нулю, а все остальные -- единицы.
\end{minipage}
\hfill
\begin{minipage}{0.2\textwidth}
\begin{verbatim}
N = 3
1 0 0
1 1 0
1 1 1
\end{verbatim}
\end{minipage}

\subsubsection*{} \noindent
\begin{minipage}{0.75\textwidth}
\paragraph*{ЛР 2.2.1.7} 
Даны натуральные числа $m,n \le 10$. Получить матрицу $A_{m \times n}$ из последовательно идущих по строкам справа налево и сверху вниз квадратов первых $mn$ натуральных чисел.
\end{minipage}
\hfill
\begin{minipage}{0.2\textwidth}
\begin{verbatim}
M = 3 N = 4
16 9 4 1
64 49 36 25
144 121 100 81
\end{verbatim}
\end{minipage}

\subsubsection*{} \noindent
\begin{minipage}{0.75\textwidth}
\paragraph*{ЛР 2.2.1.8}
Даны натуральные числа $m,n \le 10 и p \le mn$. Получить матрицу $A_{m \times n}$, заполненную последовательно идущими по строками справа налево и сверху вниз степенями 2, начиная с $2^0 = 1$.
\end{minipage}
\hfill
\begin{minipage}{0.2\textwidth}
\begin{verbatim}
M = 3 N = 4
8 4 2 1
128 64 32 16
2048 1024 512 256
\end{verbatim}
\end{minipage}

\clearpage
\paragraph*{ЛР 2.2.2.1} Дана матрица действительных чисел $B_{m \times n}$ ($m \le 10, n \le 15$) и натуральное число $p \le m$. В матрице $B$ поменять местами строку с максимальной суммой элементов (если таких строк несколько -- выбрать первую) и строку с номером $p$.
\paragraph*{ЛР 2.2.2.2} Дана матрица $B_{m \times n}$ ($m \le 10, n \le 15$), состоящая из нулей 0 и единиц 1. Удалить из матрицы $B$ все столбцы, целиком состоящие из нулей.
\paragraph*{ЛР 2.2.2.3} Дана матрица $B_{m \times n}$ ($m \le 10, n \le 15$), состоящая из нулей 0 и единиц 1. Удалить из матрицы $B$ все строки, содержащие хотя бы один нуль.
\paragraph*{ЛР 2.2.2.4} Дана матрица целых чисел $B_{m \times n}$ ($m \le 10, n \le 15$). После каждой строки, в которой сумма элементов равна нулю, дописать строку, целиком состоящую из нулей.
\paragraph*{ЛР 2.2.2.5} Дана матрица действительных чисел $B_{m \times n}$ ($m \le 10, n \le 15$), в которой хотя бы один столбец состоит из всех нулей. Поменять местами первый найденный столбец с максимальной суммой элементов и последний столбец (если таких несколько), в котором все элементы равны нулю.
\paragraph*{ЛР 2.2.2.6} Дана матрица действительных чисел $B_{m \times n}$ ($m \le 10, n \le 15$). Удалить из матрицы $B$ все строки, которые не содержат нулей, а затем прибавить к каждому элементу полученной матрицы его удвоенное абсолютное значение.

\end{document}