\documentclass[12pt,a4paper]{report}

\usepackage[left=2cm,right=2cm,top=3cm,bottom=3cm,bindingoffset=0cm,pdftex]{geometry}
\usepackage[utf8]{inputenc}
\usepackage[russian]{babel}
\usepackage{indentfirst}

\usepackage{color}
\usepackage{listings}
\usepackage{amsmath}
%\usepackage{textcomp} % Для значка градуса
\usepackage{amssymb}  % Для значка треугольника
\usepackage{amsmath}

\begin{document}
\parindent=1cm
\pagestyle{empty}

\lstset{ language=Pascal, basicstyle=\small\ttfamily, numbers=left, numberstyle=\tiny, stepnumber=1, numbersep=5pt, extendedchars=\true, showstringspaces=false, breakatwhitespace=true, frame=single, keepspaces=true }
\clearpage
theory

\clearpage
\clearpage
\subsubsection*{ДЗ №2.1}
\paragraph*{ДЗ 2.1.1} Дано натуральное число $n \le 20$ и последовательность целых чисел $a_1, ..., a_n$. Если знаки чисел чередуются ($+,-,+,-, ...$ или $-,+,-,+, ....$), то найти количество отрицательных чисел; иначе, вычислить произведение всех чисел последовательности.
\paragraph*{ДЗ 2.1.2} Дано натуральное число $n \le 20$ и последовательность целых чисел $a_1, ..., a_n$. Если все числа, стоящие на чётных позициях $a_2, a_4, a_6, ...$, --- чётные, то найти сумму всех чисел последовательности; иначе, вычислить минимум среди чисел, стоящих на нечётных позициях, $min(a_1, a_3, ...)$.
\paragraph*{ДЗ 2.1.3} Дано натуральное число $n \le 20$ и последовательность целых чисел $a_1, ..., a_n$. Если $a_1 > a_2 > ... > a_n$, то вывести среднее геометрическое чисел последовательности; иначе, найти минимум $min(a_1, ..., a_n)$.
\paragraph*{ДЗ 2.1.4} Дано натуральное число $n \le 20$ и последовательность целых чисел $a_1, ..., a_n$. Если $a_1 \le a_2 \le ... \le a_n$, то вывести среднее арифметическое чисел последовательности; иначе, найти максимум $max(a_1, ..., a_n)$.
\paragraph*{ДЗ 2.1.5} Дано натуральное число $n \le 20$ и последовательность целых чисел $a_1, ..., a_n$. Найти количество нечётных чисел в последовательности. Условный оператор if .. then не использовать.
\paragraph*{ДЗ 2.1.6} Дано натуральное число $n \le 20$ и последовательность целых чисел $a_1, ..., a_n$. Удалить из последовательности все числа $a_i$ со значением $max(a_1, ..., a_n)$, используя не более двух проходов по последовательности.
\paragraph*{ДЗ 2.1.7} Даны натуральные числа $n \le 20, m \le n$ и последовательность целых чисел $a_1, ..., a_n$. Требуется домножить все члены подпоследовательности $a_1, a_2, ..., a_{m}$ на квадрат её наименьшего члена, если $a_1 \ge 0$, и на квадрат её наибольшего члена, если $a_1 < 0$.
\paragraph*{ДЗ 2.1.8}
\paragraph*{ДЗ 2.1.9}
\paragraph*{ДЗ 2.1.X}

\clearpage
\subsubsection*{ДЗ №2.2}
\paragraph*{ДЗ 2.2.1}
\paragraph*{ДЗ 2.2.2}
\paragraph*{ДЗ 2.2.3}
\paragraph*{ДЗ 2.2.4}
\paragraph*{ДЗ 2.2.5}
\paragraph*{ДЗ 2.2.6}
\paragraph*{ДЗ 2.2.7}
\paragraph*{ДЗ 2.2.8}
\paragraph*{ДЗ 2.2.9}
\paragraph*{ДЗ 2.2.X}

\clearpage
\subsubsection*{ЛР №2.1}
% получить чиселки
\paragraph*{ЛР 2.1.1.1} Дано натуральное число $n \le 20$ и последовательность целых чисел $a_1, ..., a_n$. Вычислить среднее арифметическое чисел последовательности и количество чисел, которые превышают его.
\paragraph*{ЛР 2.1.1.2} Дано натуральное число $n \le 20$, последовательность целых чисел $x_1, ..., x_n$ и последовательность действительных чисел $b_1, ..., b_n$. Вычислить $\frac { x_1 b_1 + x_2 b_2 + ... + x_n b_n} { x_1 + x_2 + ... + x_n }$ и найти минимум $min(x_1 b_1, x_2 b_3, ..., x_{n-1} b_{n-1})$.
\paragraph*{ЛР 2.1.1.3} Дано натуральное число $n \le 20$ и последовательность целых чисел $a_1, ..., a_n$. Вычислить минимум $min(a_1, ..., a_n)$, максимум $max(a_1, ..., a_n)$, среднее арифметическое и среднее геометрическое чисел последовательности.
\paragraph*{ЛР 2.1.1.4} Дано натуральное число $n \le 20$ и две последовательности целых чисел $a_1, ..., a_n$ и $b_1, ..., b_n$. Вычислить $(a_{n-1} + b_2)(a_{n-3} + b_4) ... (a_1 + b_{n})$ и найти максимум $max(a_1, ..., a_n)$. 
\paragraph*{ЛР 2.1.1.5} Дано натуральное число $n \le 20$ и две последовательности целых чисел $a_1, ..., a_n$ и $b_1, ..., b_n$. Вычислить $(a_1 + b_n)(a_2 + b_{n-1}) ... (a_n + b_1)$ и найти минимум $min(b_1, ..., b_n)$.

% получить новую последовательность
\paragraph*{ЛР 2.1.2.1} Дана последовательность действительных чисел $a_1, ..., a_{20}$. \\
Получить новую последовательность $a_{20}, a_{11}, a_{19}, a_{10}, ..., a_{10}, a_1$.
\paragraph*{ЛР 2.1.2.2} Дана последовательность действительных чисел $a_1, ..., a_{20}$. \\
Получить новую последовательность  $a_1, a_3, ..., a_{19}, a_2, a_4, ..., a_{20}$.
\paragraph*{ЛР 2.1.2.3} Дана последовательность действительных чисел $a_1, ..., a_{20}$. \\
Получить новую последовательность  $a_1, a_{11}, a_3, a_{13}, ..., a_9, a_{19}$.
\paragraph*{ЛР 2.1.2.4} Дана последовательность действительных чисел $a_1, ..., a_{20}$. \\
Получить новую последовательность  $a_{12}, a_2, a_{14}, a_4, ..., a_{20}, a_{10}$.
\paragraph*{ЛР 2.1.2.5} Дана последовательность действительных чисел $a_1, ..., a_{20}$. \\
Получить новую последовательность  $a_1, a_{20}, a_3, a_{18}, ..., a_{19}, a_2$.

\clearpage
% удалить/вставить в одном и том же массиве
\paragraph*{ЛР 2.1.3.1} Дано натуральное число $n \le 20$ и последовательность целых чисел $x_1, ..., x_n$. Удалить из последовательности все нули $0$.
\paragraph*{ЛР 2.1.3.2} Даны натуральные числа $n \le 20, p \le 100$ и последовательность целых чисел $x_1, ..., x_n$. После каждого  числа последовательности $x_i$, кратного $p$, вставить произведение $x_i p$.
\paragraph*{ЛР 2.1.3.3} Даны натуральные числа $n \le 20, p \le 100$ и последовательность целых чисел $x_1, ..., x_n$. Удалить из последовательности все числа $x_i$, кратные $p$.
\paragraph*{ЛР 2.1.3.4} Дано натуральное число $n \le 20$ и последовательность целых чисел $x_1, ..., x_n$. После каждого отрицательного числа $x_i<0$ вставить его модуль $|x_i|$.
\paragraph*{ЛР 2.1.3.5} Дано натуральное число $n \le 20$ и последовательность целых чисел $x_1, ..., x_n$. Оставить в последовательности только неотрицательные числа $x_i \ge 0$, а все остальные удалить.

\clearpage
\subsubsection*{ЛР №2.2}
\paragraph*{ЛР 2.2.1.1}
\paragraph*{ЛР 2.2.1.2}
\paragraph*{ЛР 2.2.1.3}
\paragraph*{ЛР 2.2.1.4}
\paragraph*{ЛР 2.2.1.5}
\paragraph*{ЛР 2.2.2.1}
\paragraph*{ЛР 2.2.2.2}
\paragraph*{ЛР 2.2.2.3}
\paragraph*{ЛР 2.2.2.4}
\paragraph*{ЛР 2.2.2.5}



\end{document}