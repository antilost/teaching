\documentclass[12pt,a4paper]{report}

\usepackage[left=2cm,right=2cm,top=3cm,bottom=3cm,bindingoffset=0cm,pdftex]{geometry}
\usepackage[utf8]{inputenc}
\usepackage[russian]{babel}
\usepackage{indentfirst}

\usepackage{color}
\usepackage{listings}
\usepackage{amsmath}
%\usepackage{textcomp} % Для значка градуса
\usepackage{amssymb}  % Для значка треугольника
\usepackage{amsmath}

\begin{document}
\parindent=1cm
\pagestyle{empty}

\lstset{ language=Pascal, basicstyle=\small\ttfamily, numbers=left, numberstyle=\tiny, stepnumber=1, numbersep=5pt, extendedchars=\true, showstringspaces=false, breakatwhitespace=true, frame=single, keepspaces=true }

\clearpage

\subsubsection{ 3.3.1.1. } Даны шесть действительных чисел $x_A, y_A, x_B, y_B, x_C, y_C$ -- координаты $x, y$ вершин треугольника A, B и C, соответственно. Найти площадь треугольника.

В программе описать запись Point для представления точки на плоскости и запись Triangle для представления треугольника, заданного тремя точками (стуктура Triangle должна использовать Point).

Для вычисления площади треугольника реализовать функцию \texttt{function Square(T: Triangle): Real;}.

Для вычисления длины стороны треугольника реализовать процедуру \texttt{procedure Distance(P1: Point; P2: Point; var D: Real);},
которая вычисляет расстояние между точками P1, P2 и записывает результат в переменную D.


\subsubsection{ 3.3.1.2. } Даны четыре действительных числа $x, y, r_1, r_2$, определяющие кольцо на плоскости с центром в точке $x, y$, внутренним радиусом $r1$ и внешним радиусом $r2$; и $n < 10$ пар действительных чисел $x_i, y_i$ -- координат точек на плоскости. Вычислить количество точек, которые находятся в плоскости кольца (заключены в область между внешней и внутренней окружностями кольца).

В программе описать запись Point для представления точки на плоскости и запись Ring для представления кольца, заданного точкой центра и двумя радиусами.

Для определения того, что точка принадлежит кольцу, написать функцию \texttt{function IsPointInsideOfRing(P: Point; R: Ring): Boolean;}, которая будет возвращать texttt{true}, если точка \texttt{P} принадлежит кольцу, параметры которого указаны в записи \texttt{R}, и \texttt{false}, иначе.

Для вычисления расстояния между двумя точками \texttt{A} и \texttt{B} написать функцию \texttt{function GetDistance(A, B: Point): Real;}.


\subsubsection{ 3.3.2.1. }
Написать программу для обработки записей с данными о сдаче сессии. Каждая запись в программе описывается в виде:
\begin{verbatim}
Exams = record
  Name: String[30];
  Math, Physics, History: Byte;
end;
\end{verbatim}
Поле \texttt{Name} -- имя студента (до 30 символов), \texttt{Math}, \texttt{Physics} и \texttt{History} -- оценки за экзамен по математике, физике и истории, соответственно (целые числа от 2 до 5).

В программе формируется массив из $N \le 20$ записей (данные указываются пользователем). Упорядочить записи по убыванию среднего балла за три экзамена.

В описание записи по необходимости можно добавлять новые поля.


\subsubsection{ 3.3.2.2. }
Написать программу для обработки записей с данными о сдаче сессии. Каждая запись в программе должна быть представлена в виде структуры вида:
\begin{verbatim}
Exams = record
  Name: String[30];
  Math, Physics, History: Byte;
end;
\end{verbatim}
Поле \texttt{Name} -- имя студента (до 30 символов), \texttt{Math}, \texttt{Physics} и \texttt{History} -- оценки за экзамен по математике, физике и истории, соответственно (целые числа от 2 до 5).
В программе формируется массив из $N \le 20$ записей (данные указываются пользователем). Упорядочить записи по убыванию среднего балла.



\clearpage
\subsection*{Рубежный контроль №3}
\subsubsection*{Вариант №1}
\paragraph*{Задача 1.} Дана строка, состоящая из слов, разделенных пробелами (каждое слово -- последовательность подряд идущих букв латинского алфавита) и натуральное число $n$. Заменить все слова длины $n$ на знак вопроса "?".
\paragraph*{Задача 2.} Задать массив из $n \le 10$ записей TParal с параметрами параллелепипедов $a, b$ и $h$, где $a, b$ -- ширина и длина оснований, а $h$ -- высота. Вывести массив паралеллепипедов (значения параметров), упорядоченный по убыванию объёма.
Для вычисления объёма реализовать и использовать в основной программе функцию \texttt{function Volume(P: TParal): Real}.
\\
\subsection*{Рубежный контроль №3}
\subsubsection*{Вариант №2}
\paragraph*{Задача 1.} Дана строка, состоящая из слов, разделенных пробелами (каждое слово -- последовательность подряд идущих букв латинского алфавита). Вывести все слова из исходной строки в порядке убывания длины (сначала самое длинное слово, а в конце -- самое короткое). Для сортировки реализовать и использовать в основной программе процедру \texttt{procedure Sort(var Words: StrArr)}, в которой \texttt{StrArr} -- объявленный в блоке \texttt{type} тип, описывающий массив строк.
\paragraph*{Задача 2.} Заданы параметры шара и цилиндра $R, r, h$; $R$ -- радиус шара, $r, h$ -- радиус основания и высота цилиндра, соответственно. Вывести название фигуры, объём которой больше. Для вычисления объёма шара и объёма цилиндра реализовать и использовать в основной программе функции \texttt{function SphereVolume(R: Real): Real} и \texttt{function CylinderVolume(R: Real; H: Real): Real}.

\subsection*{Рубежный контроль №3}
\subsubsection*{Вариант №2}
\paragraph*{Задача 1.} Проверить, что заданное слово, состоящее из букв латинского алфавита, является палиндромом. Для проверки написать функцию \texttt{function IsPalindrome(S: String): Boolean}, которая принимает на вход исходное слово, заданное в строке S, и возвращает \texttt{true}, если слово является палиндромом и \texttt{false}, иначе.
Палиндромами называют слова, в которых порядок букв одинаковый независимо от порядка прочтения слева-направо или справа-налево. Например, \textit{radar}, \textit{RSFSR} -- палиндромы.

\paragraph*{Задача 2.} Заданы параметры пирамиды и конуса $a, h, r, H$; $a, h$ -- сторона квадрата-основания и высота пирамиды, а $r, H$ -- радиус основания и высота конуса. Вывести название фигуры с меньшим объёмом. Для вычисления объёма фигур реализовать и использовать в основной программе функции \texttt{function PyramidVolume(A: Real; H: Real): Real} и \texttt{function ConeVolume(R: Real; H: Real): Real}.




\clearpage
\subsection*{Рубежный контроль №2}
\subsubsection*{Вариант №13}
\paragraph*{Задача 1.} Задан массив из $n \le 10$ целых чисел. Заменить все максимумы на минимумы, и наоборот.
Пример, исходная последовательность: \textit{22, 10, 76, 15, 10, 10, 14, 76, 33, 51}, результат: \textit{22, 76, 10, 15, 76, 76, 14, 10, 33, 51}.
\paragraph*{Задача 2.} Дана матрица $A_{n \times m}, n \le 10, m \le 15$. Удалить строки, в которых сумма элементов, стоящих в столбцах с нечётными номерами, равна нулю.
\subsubsection*{Вариант №42}
\paragraph*{Задача 1.} Даны два массива из $n \le 10$ и $m \le 10$ целых чисел, упорядоченных по возрастанию. Получить новый массив, состоящий из чисел, встречающихся в обоих исходных массивах, также упорядоченный по возрастанию. В исходных массивах числа не повторяются.
\paragraph*{Задача 2.} Дана матрица $A_{n \times m}, n \le 10, m \le 15$. Поменять местами столбцы с максимальной и минимальной суммой элементов.

\end{document}
