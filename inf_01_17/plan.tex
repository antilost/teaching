\documentclass[12pt,a4paper]{report}

\usepackage[left=2cm,right=2cm,top=3cm,bottom=3cm,bindingoffset=0cm,pdftex]{geometry}
\usepackage[utf8]{inputenc}
\usepackage[russian]{babel}
\usepackage{indentfirst}

\usepackage{color}
\usepackage{listings}
\usepackage{amsmath}
\usepackage{textcomp} % Для значка градуса
\usepackage{amssymb}  % Для значка треугольника

\begin{document}
\parindent=1cm
\pagestyle{empty}

\lstset{ language=Pascal, basicstyle=\small\ttfamily, numbers=left, numberstyle=\tiny, stepnumber=1, numbersep=5pt, extendedchars=\true, showstringspaces=false, breakatwhitespace=true, frame=single, keepspaces=true }


\clearpage
\section*{План семестра}
\subsection*{Модуль 1}
\begin{itemize}
\item ДЗ 1.1 (1)
\item ДЗ 1.2 (1)
\item ЛР 1.1 (2)
\item ЛР 1.2 (2)
\item РК 1.1 (3)
\item РК 1.2 (3)
\end{itemize}

\clearpage


\paragraph*{ДЗ 1.1.1}
Три сопротивления $R_1$, $R_2$, $R_3$ соединены параллельно. Найти сопротивление соединения.

\paragraph*{ДЗ 1.1.2}
Определить время падения камня на поверхность земли с высоты $h$.


\paragraph*{ДЗ 1.1.3}
Треугольник на плоскости задан координатами своих вершин. Найти площадь этого треугольника.

\paragraph*{ДЗ 1.1.4}
Дана сторона равностороннего треугольника. Найти площадь этого треугольника.

\paragraph*{ДЗ 1.1.5}
Даны катеты прямоугольного треугольника. Найти его гипотенузу, периметр и площадь.

\paragraph*{ДЗ 1.1.6}
Даны катеты прямоугольного треугольника. Найти величину наименьшего угла в градусах.






\paragraph*{ДЗ 1.1.1}
Даны два действительных положительных числа. Найти среднее арифметическое и среднее геометрическое этих чисел.

\paragraph*{ДЗ 1.1.2}
Даны катеты прямоугольного треугольника. Найти его гипотенузу и площадь.









\end{document}