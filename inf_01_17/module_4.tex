\documentclass[12pt,a4paper]{report}

\usepackage[left=1cm,right=1cm,top=1cm,bottom=2cm,bindingoffset=0cm,pdftex]{geometry}
\usepackage[utf8]{inputenc}
\usepackage[russian]{babel}
\usepackage{indentfirst}

\usepackage{color}
\usepackage{listings}
\usepackage{amsmath}
%\usepackage{textcomp} % Для значка градуса
\usepackage{amssymb}  % Для значка треугольника
\usepackage{amsmath}

\begin{document}
\parindent=1cm
\pagestyle{empty}

\lstset{ language=Pascal, basicstyle=\small\ttfamily, numbers=left, numberstyle=\tiny, stepnumber=1, numbersep=5pt, extendedchars=\true, showstringspaces=false, breakatwhitespace=true, frame=single, keepspaces=true }

\clearpage
\subsection*{Что нужно сделать?}
\begin{enumerate}
\item Выбрать тематику: один из трёх больших блоков задач;
\item Выбрать варианты задач: один из шести вариантов для первой задачи, один из трёх для второй и один из двух для третьей. Первая задача должна быть у каждого своя; каждый вариант задачи №2 могут делать не более двух человек; каждый вариант задачи №3 могут делать не более трёх человек;
\item Решить задачу №0, обязательно в самом начале;
\item Решить задачи №1, №2, и №3 (в любом порядке).
\end{enumerate}

Задача №0 -- это основа вашей программы. Здесь нужно написать программу, в которой вводится имя обрабатываемого типизированного файла, и в бесконечном цикле меню пользователь может выбрать одно из трёх доступных действий: запись $N>0$ новых записей в файл (перезапись существующего файла или создание нового), вывод всех данных из файла (чтение) и выход из программы -- завершение цикла.

В задачах №1 и №2 в созданную ранее в задаче №0 программу нужно добавить новую процедуру, которая будет решать указанную проблему. Новая процедура может быть вызвана по выбору пользователя из меню, так же как и базовые действия из задачи №0.

В задаче №3 в программу также нужно добавить новую процедуру, вызываемую по выбору пользователя, но и иногда не только её.


\subsection*{1. Финансовая статистика предприятия}
Здесь нужно написать программу для обработки файла записей, каждая из которых хранит данные о финансовой деятельности предприятия за определённый год: общий размер доходов и расходов. В общем случае значения всех полей могут повторяться, если в задаче не указано иное.

Структура записи
\begin{verbatim}
FinStat = record
  Year: Integer; // Год
  Income: Real;  // Сумма доходов
  Cost: Real;    // Сумма расходов
end;
\end{verbatim}

\subsubsection{Задача 1.0.} Реализовать общую для всех тем функциональность --- ввод имени файла, цикл меню с выбором одного из трёх базовых действий: запись, чтение файла и выход из программы.

\subsubsection{Задача 1.1.}
\paragraph{1.1.1.} Вывести записи из файла в формате: год \texttt{Year}, процентное соотношение суммы расходов к сумме доходов $ \frac {Cost} {Income} * 100 \% $.
\paragraph{1.1.2.} Найти в файле запись с минимальным значением прибыли (прибыль равна разнице доходов \texttt{Income} и расходов \texttt{Cost}). 
\paragraph{1.1.3.} Вычислить суммарную прибыль по всем записям в файле (прибыль равна разнице \texttt{Income} и расходов \texttt{Cost}).
\paragraph{1.1.4.} Добавить в конец файла одну новую запись (значения полей вводятся пользователем).
\paragraph{1.1.5.} Найти количество записей в файле, когда расходы \texttt{Cost} превышали доходы \texttt{Income}.
\paragraph{1.1.6.} Вычислить прибыль для заданного года \texttt{Year} (значение вводится пользователем), прибыль равна разнице \texttt{Income} и расходов \texttt{Cost}. Для одного и того же года \texttt{Year} в файле может быть несколько записей.

\subsubsection{Задача 1.2.}
\paragraph{1.2.1.} Изменить порядок следования записей в файле на противоположный (первая запись становится последней, вторая -- предпоследней и т.д.).
\paragraph{1.2.2.} Удалить из файла все записи с указанным значением года \texttt{Year}.
\paragraph{1.2.3.} Заменить каждую запись с ненулевым значением расходов на две: в первой хранятся доходы, а расходы равны нулю, а во второй хранятся расходы, а доходы равны нулю; новая запись добавляется в файл сразу после предыдущей.

\textbf{Пример.} Формат: $(x;y)$, $x$ -- число доходов, $y$ -- число расходов.
\begin{verbatim}
До обработки:    (4;0), (6;4),        (2;0), (9;5)
После обработки: (4;0), (6;0), (0;4), (2;0), (9;0), (0;5)
\end{verbatim}


\subsubsection{Задача 1.3.}
\paragraph{1.3.1. Объединение файлов.} Добавить две процедуры: \texttt{SortFile} и \texttt{MergeFiles}.

Процедура \texttt{SortFile} сортирует записи в обрабатываемом файле в порядке возрастания значений Year, и её вызов должен производиться каждый раз, когда содержимое файла могло поменяться (добавление, замена или удаление компонент файла), чтобы записи в файле всегда были упорядочены.

Процедура \texttt{MergeFiles} вызывается по выбору пользователя из цикла меню. В процедуре \texttt{MergeFiles} задаётся имя второго типизированного файла, так же состоящего из записей FinStat (созданного в программе ранее, при другом запуске); процедура должна по ключу \texttt{Year} объединить содержимое файлов, просуммировав \texttt{Cost} и \texttt{Income}, соответственно.

Пусть $F$ -- текущий обрабатываемый файл (имя задано при запуске программы), $B$ -- файл, указанный в процедуре \texttt{MergeFiles}.
Если запись с годом $Y$ есть только в файле $F$ -- она остаётся в нём без изменений; если запись с годом $Y$ есть только в файле $B$, то она добавляется к файлу $F$; если запись с годом $Y$ есть и в файле $F$, и в файле $B$, то к значениям \texttt{Cost} и \texttt{Income} записи из файла $F$ добавляются значения \texttt{Cost} и \texttt{Income}, соответственно, из записи файла $B$

\textbf{Пример.} Записи в файлах $F$, $B$ до и после обработки. Формат: $year (x;y)$, $year$ -- год, ключ для слияния, $x$ -- число доходов, $y$ -- число расходов.
\begin{verbatim}
F: 2014 (4;3), 2015 (6;4), 2016 (5;7)
B: 2014 (1;0),             2016 (2;0), 2017 (6;3)
...обработка...
F: 2014 (5;3), 2015 (6;4), 2016 (7;7), 2017 (6;3)
\end{verbatim}


\paragraph{1.3.2. Удаление дубликатов.} Добавить две процедуры: \texttt{SortFile} и \texttt{RemoveDuplicates}.

Процедура \texttt{SortFile} сортирует записи в обрабатываемом файле в порядке возрастания значений \texttt{Year}, и её вызов должен производиться каждый раз, когда содержимое файла могло поменяться (добавление, замена или удаление компонент файла), чтобы записи в файле всегда были упорядочены.

Процедура \texttt{RemoveDuplicates} вызывается по выбору пользователя из цикла меню. В процедуре \texttt{RemoveDuplicates} производится удаление дубликатов записей по ключу \texttt{Year} из обрабатываемого файла, при этом значения доходов и расходов, соответственно, из записей с одинаковым значением \texttt{Year} суммируются. (После вызовов \texttt{SortFile} файл всегда будет упорядоченым -- используйте это!)

\textbf{Пример.} Записи в файле до и после обработки. Формат: $year (x;y)$, $year$ -- год, ключ для слияния, $x$ -- число доходов, $y$ -- число расходов.
\begin{verbatim}
2014 (3; 4), 2015 (1;0), 2015(0;5), 2016(4;5), 2016(2;0), 2018(0;1)
2014 (3; 4), 2015 (1;5), 2016(6;5), 2018(0;1)
\end{verbatim}


\subsection*{2. Интернет-магазин}
Данная тематика описывает программу для обработки записей с информацией о продуктах (товарах) некоторого Интернет-магазина.

Структура записи
\begin{verbatim}
Product = record
  Name: String[30]; // Наименование продукта
  Price: Real;      // Цена за единицу (руб.)
  Sold: Integer;    // Количество проданных единиц
  Balance: Integer; // Остаток на складе
end;
\end{verbatim}

\subsubsection{Задача 2.0.} Реализовать общую для всех тем функциональность --- ввод имени файла, цикл меню с выбором одного из трёх базовых действий: запись, чтение файла и выход из программы.

\subsubsection{Задача 2.1.}
\paragraph{2.1.1.} Вычислить суммарную стоимость \texttt{Price} всех имеющихся на складе товаров \texttt{Balance} > 0.
\paragraph{2.1.2.} Найти товар с минимальной ценой и товар с максимальной ценой \texttt{Price}.
\paragraph{2.1.3.} Вычислить сумму доходов магазина от всех проданных товаров (сумма доходов одного товара равна произведению цены \texttt{Price} и количества проданных единиц \texttt{Sold}).
\paragraph{2.1.4.} Найти товар, который лучше всего продаётся: запись с наибольшим значением $\frac { Sold } { Sold + Balance } $.
\paragraph{2.1.5.} Вывести записи из файла в формате: наименование продукта, цена в рублях, цена в долларах по заданному курсу (указывается сколько рублей стоит один доллар).
\paragraph{2.1.6.} Найти все данные о товаре по заданному наименованию \texttt{Name}.

\subsubsection*{Задача 2.2.}
\paragraph{2.2.1.} Удалить все товары с нулевым остатком \texttt{Balance} = 0.
\paragraph{2.2.2.} Найти наименования продуктов \texttt{Name}, встречающиеся несколько раз; показать наименование и число вхождений.
\paragraph{2.2.3.} Вводятся данные новой записи; если в файле нет записей с таким же наименованием \texttt{Name}, как у новой записи, то добавить запись в конец, иначе обновить первую найденную в файле запись с таким же \texttt{Name}.

\subsubsection*{Задача 2.3.}
\paragraph{2.3.1. Разделение файла.} Добавить процедуру \texttt{DivideFile}, в которой вводятся имена новых файлов $A, B, C$ и границы диапазонов цен $0 < P_A < P_B < P_C$. Процедура должна удалить из текущего обрабатываемого файла записи, цена которых подходит под соответствующий диапазон цен, и записать их в новый файл.

Если цена $P_A \le Price < P_B$, то запись помещается в файл $A$; если цена $P_B \le Price < P_C$, то запись помещается в файл $B$; если цена $P_C \le Price$, то запись помещается в файл $C$; иначе, когда выполняется $0 \le Price < P_A$, запись остаётся в текущем файле.

\paragraph{2.3.2. Топ.} Составить список из $N>0$ отчётных лет по убыванию значения прибыли. Года в файле могут повторяться, и значения \texttt{Income}, \texttt{Cost} для соответствующих лет должны быть просуммированы



\subsection*{3. Библиотека}
В задачах данного блока типизированный файл состоит из записей с информацией о книгах: библиотека.
Структура записи
\begin{verbatim}
Book = record
  ID: Integer;         // Уникальный идентификатор книги
  Name: String[30];    // Название книги
  Author: String[20];  // Автор книги
  Year: Integer;       // Год издания
  PagesCount: Integer; // Число страниц
end;
\end{verbatim}

\subsubsection{Задача 3.0.} Реализовать общую для всех тем функциональность --- ввод имени файла, цикл меню с выбором одного из трёх базовых действий: запись, чтение файла и выход из программы.
\subsubsection{Задача 3.1.}
\paragraph{3.1.1.} Вычислить минимальный и максимальный года издания \texttt{Year} для всех книг указанного автора \texttt{Author}.
\paragraph{3.1.2.} Проверить, что идентификаторы книг \texttt{ID} в файле не повторяются.
\paragraph{3.1.3.} Найти количество всех книг, которые были выпущены в указанном году \texttt{Year}.
\paragraph{3.1.4.} Выяснить, сколько страниц \texttt{PagesCount} потребуется для издания всех книг указанного автора \texttt{Author}.
\paragraph{3.1.5.} Вывести \texttt{ID}, название \texttt{Name} и автора \texttt{Author} книг, изданных в указанный промежуток лет \texttt{Year} $\in [a; b]$.
\paragraph{3.1.6.} Заданы два автора \texttt{Author}. Выяснить, кто написал больше книг.

\subsubsection*{Задача 2.2.}
\paragraph{3.2.1.} Для указанного автора найти две книги с наименьшим числом страниц (если книг нет -- вывести соответствующее сообщение, если книга одна -- вывести только её).
\paragraph{3.2.2.} Удалить из файла все книги указанного автора \texttt{Author}.
\paragraph{3.2.2.} Скопировать все записи с книгами указанного автора \texttt{Author} в новый файл. Имя файла вводится пользователем.


\subsubsection*{Задача 3.3.}
\paragraph{3.3.1. Бинарный поиск.} Добавить две процедуры: \texttt{SortFile} и \texttt{BinSearch}.

Процедура \texttt{SortFile} сортирует записи в обрабатываемом файле в порядке возрастания значений \texttt{ID}, и её вызов должен производиться каждый раз, когда содержимое файла могло поменяться (добавление, замена или удаление компонент файла), чтобы записи в файле всегда были упорядочены.

Процедура \texttt{BinSearch} вызывается по выбору пользователя из цикла меню. В процедуре \texttt{BinSearch} производится нахождение записи в файле при помощи бинарного поиска (двоичный поиск, метод деления пополам): задано значение \texttt{ID} книги, которую нужно вывести. Заданное значение сравнивается с записью в середине интервала поиска, и в зависимости от результата сравнения выбирается следующий интервал поиска.

После вызовов \texttt{SortFile} файл всегда будет упорядоченым по значению \texttt{ID} -- это необходимое условие для работы бинарного поиска.

\paragraph{3.3.2. Перестановка.} Добавить процедуру \texttt{MoveToTheEnd}, которая переставляет все книги указанного автора \texttt{Author} в конец файла, и устанавливает в соответствующих записях \texttt{PagesCount} = 0.



\end{document}
