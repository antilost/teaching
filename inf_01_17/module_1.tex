\documentclass[12pt,a4paper]{report}

\usepackage[left=2cm,right=2cm,top=3cm,bottom=3cm,bindingoffset=0cm,pdftex]{geometry}
\usepackage[utf8]{inputenc}
\usepackage[russian]{babel}
\usepackage{indentfirst}

\usepackage{color}
\usepackage{listings}
\usepackage{amsmath}
\usepackage{textcomp} % Для значка градуса \textdegree
\usepackage{amssymb}  % Для значка треугольника
\usepackage{amsmath}

\begin{document}
\parindent=1cm
\pagestyle{empty}

\lstset{ language=Pascal, basicstyle=\small\ttfamily, numbers=left, numberstyle=\tiny, stepnumber=1, numbersep=5pt, extendedchars=\true, showstringspaces=false, breakatwhitespace=true, frame=single, keepspaces=true }

\clearpage
Знакомство с системой программирования Delphi (консольный режим)
Программа (программный код) представляет собой последовательность символов, к числу которых относятся буквы, цифры, знаки препинания, знаки операций.


\subsubsection*{Структура программы}
Простейшая программа
\begin{verbatim}
program HelloWorld;

begin
  WriteLn('Hello, world!');
  ReadLn;
end.
\end{verbatim}


\subsubsection*{Основные стандартные типы данных}
Integer
Real
Char
String

\subsubsection*{Оператор присваивания}
=

\subsubsection*{Выражения}

\subsubsection*{Процедуры ввода-вывода}



\clearpage
\subsubsection*{Разработка программ линейной структуры.}
Программа вычисления площади круга по заданному значению радиуса. 
\begin{verbatim}
program CircleSquare;

{
    Вычисление площади круга по заданному радиусу R
}

const
  MY_PI = 3.14; // Константа MY_PI, её значение не может быть изменено

var
  R: Real; // Переменная для значения радиуса
  S: Real; // Площадь круга

begin
    Write('Input radius: '); // Вывод подсказки для пользователя (что вводим)
    ReadLn(R);               // Ввод с клавиатуры значения радиуса 

    S := MY_PI * R * R;      // Вычисление площади круга по известной формуле

    WriteLn('Circle square is ', S:0:2); // Вывод результата

    ReadLn; // Ожидание нажатия любой клавиши,
            // чтоб программа не завершилась раньше времени
            // и показала результат
end.
\end{verbatim}


\clearpage
Программа для вычисления квадратного уравнения
\begin{verbatim}
program SquareEquation;

var
  A, B, C, D, X1, X2: Real;
begin
  WriteLn('Введите значения коэффициентов');
  Write('A: '); ReadLn(A);
  
  if A = 0 then
  begin
    // Код внутри данного блока begin-end
    // будет выполнен только если A = 0
    WriteLn('A = 0');
    ReadLn;
    Exit; // Команда выхода из программы
  end;
  
  Write('B: '); ReadLn(B);
  Write('C: '); ReadLn(C);

  D := Sqr(B) - 4 * A * C; // Вычисление дискриминанта

  if D < 0 then
  begin
    // Этот код выполнится только если дискриминант будет меньше нуля
    WriteLn('Корни мнимые');
  end
  else
  begin
    // Этот блок выполнится если дискриминант не меньше 0 (больше, либо равен)
    if D = 0 then
    begin
      X1 := -B / (2 * A);
      WriteLn('Один корень, X = ', X1:0:2);
    end
    else
    begin
      X1 := (-B + Sqrt(D)) / 2 / A;
      X2 := (-B - Sqrt(D)) / 2 / A;
      WriteLn('X1 = ', X1:0:2, ', X2 = ', X2:0:2);
    end;
  end;

  ReadLn;
end.
\end{verbatim}




\clearpage
\subsubsection*{ДЗ №1.1}
\paragraph*{ДЗ 1.1.1} Треугольник $\Delta ABC$ на плоскости задан координатами своих вершин. Найти периметр и площадь этого треугольника. Определить скалярное произведение векторов $\overrightarrow{AB}$ и $\overrightarrow{BC}$.
% TODO: replace task 1.1.2 with simple one
\paragraph*{ДЗ 1.1.2} Даны целые числа $h,m$ ($0 < h \le 12, 0 < m \le 60$), указывающие момент времени "$h$ часов $m$ минут". Определить наименьшее время (число полных минут) до момента, когда часовая и минутная стрелки на циферблате совпадут.
\paragraph*{ДЗ 1.1.3} Даны три точки $A, B, C$ на плоскости, заданные координатами своих вершин. Определить, расположены ли они на одной прямой. Если нет, то вычислить угол $ABC$.
\paragraph*{ДЗ 1.1.4} Даны два числа $a, b$. Если они одновременно чётные или одновременно нечётные, то вывести абсолютное значение (модуль) разности этих чисел; иначе, вывести квадрат их произведения. Вывести наибольшее из чисел $a, b$.
\paragraph*{ДЗ 1.1.5} Даны два числа $x_1, x_2$ и функция $f(x)$, заданная в виде:
\begin{equation*}
 f(x) = 
 \begin{cases}
   e^x, x \le 0; \\
   tg(x), 0 < x \le 1; \\
   lg(x), 1 < x.
 \end{cases}
\end{equation*}
Из $x_1, x_2$ выбрать то значение, при котором значение функции наибольшее.
\paragraph*{ДЗ 1.1.6} Даны действительные числа $x, y$. Если $x$ и $y$ отрицательные, то каждое значение заменить его модулем. Если отрицательно только одно из них, то оба значения увеличить на 0.5. Во всех остальных случаях уменьшить оба значения в 10 раз.
\paragraph*{ДЗ 1.1.7} Даны действительные числа $x, y, z, x > 0, y > 0, z > 0$. Выяснить, существует ли треугольник с такими сторонами. Если существует, то вывести величину наибольшего угла (в градусах).
\paragraph*{ДЗ 1.1.8} Даны действительные числа $x_1, y_1, x_2, y_2, x_3, y_3$. Принадлежит ли начало координат треугольнику с вершинами ($x_1;y_1$), ($x_2;y_2$), ($x_3;y_3$)?
\paragraph*{ДЗ 1.1.9} Даны действительные числа $x, y$ ($x \neq y$). Меньшее из этих двух чисел заменить их полусуммой, а большее -- их удвоенным произведением. 
\paragraph*{ДЗ 1.1.10} Дана комната, представляющая собой параллелепипед с длиной $a$, шириной $b$ и высотой $h$, заданных в метрах (наличием окон, дверей и других проёмов пренебречь, считая плоскость стен непрерывной). Сколько нужно рулонов обоев для того, чтобы оклеить стены этой комнаты целиком? Каждый рулон имеет длину 10 метров и ширину $s$ метров.


\clearpage
\subsubsection*{ЛР №1.1}
\paragraph*{ЛР 1.1.1.1} Даны действительные числа $x, y, z$. Вычислить $max(x+y+z, xyz)$.
\paragraph*{ЛР 1.1.1.2} Даны действительные числа $x, y$. Вычислить z:
\begin{equation*}
 z = 
 \begin{cases}
   x - y, \text{ при } x > y, \\
   y - x + 1, \text{ в противном случае}.
 \end{cases}
\end{equation*}
\paragraph*{ЛР 1.1.1.3} Даны действительные числа $x, y, z$. Выбрать из них те, которые принадлежат интервалу $[a; b)$.
\paragraph*{ЛР 1.1.1.4} Даны действительные числа $x, y, z$. Выбрать из них те, которые не входят в интервал $[c; d)$.
\paragraph*{ЛР 1.1.1.5} Даны целые числа $a, b, c$. Выбрать из них максимальное чётное значение.

\paragraph*{ЛР 1.1.2.1} Вычислить площадь кольца с внутренним радиусом $r_1$ и внешним $r_2$.
\paragraph*{ЛР 1.1.2.2} Указаны длина гипотенузы $x$ и длина одного из катетов $y$ прямоугольного треугольника. Вычислить площадь этого треугольника.
\paragraph*{ЛР 1.1.2.3} Найти периметр треугольника, заданного координатами своих вершин.
\paragraph*{ЛР 1.1.2.4} Вычислить объём цилиндра с радиусом $r$ и высотой $h$.
\paragraph*{ЛР 1.1.2.5} Обычно пицца имеет форму круга, покрытого сыром, с краешком из теста. Допустим дана пицца радиуса $r$ с шириной краешка $c$. Вычислить, сколько процентов от всей пиццы занимает сыр.

\clearpage
\subsubsection*{ДЗ №1.2}
\paragraph*{ДЗ 1.2.1} Получить таблицу температур $t_c$ по Цельсию от 0 \textdegree до 100 \textdegree и их эквивалентов по шкале Фаренгейта $t_F$, используя для перевода формулу $t_F = \frac {9} {5} t_c + 32$.
\paragraph*{ДЗ 1.2.2} Дано натуральное число $n$. Получить все его натуральные делители.
\paragraph*{ДЗ 1.2.3} Даны натуральное число $n$ и целые числа $a_1, a_2, ..., a_n$. Получить произведение членов последовательности $a_1, a_2, ..., a_n$, которые делятся на $p$ нацело (без остатка).
\paragraph*{ДЗ 1.2.4} Дано натуральное число $n$ и действительные числа $a_1, a_2, ..., a_n$. Каждое отрицательное число из последовательности $a_1, a_2, ..., a_n$ уменьшить на 0.1, а положительное увеличить на 0.5.
\paragraph*{ДЗ 1.2.5} Дано натуральное число $n$. Вывести $L_n$ -- $n$-й член последовательности чисел Люка, вычисляемый по формуле:
\begin{equation*}
 L_i = 
 \begin{cases}
   2, i = 0; \\
   1, i = 1; \\
   L_{i-2} + L_{i-1}, i > 1. \\
 \end{cases}
\end{equation*}
\paragraph*{ДЗ 1.2.6} Дано натуральное число $n$, действительные числа $x_1 \le x_2 \le ... \le x_n$ и $a$. Из чисел $x_1, ... x_n$ и числа $a$ получить новую последовательность $y_1, ... y_{n+1}$, такую, что $y_1 \le y_2 \le ... \le y_n$.
\paragraph*{ДЗ 1.2.7} Задан круг радиуса $R$ с центром в точке $(x_r; y_r)$ и $n$ точек $(x_1; y_1), ... (x_n; y_n)$. Вывести количество точек, которые находятся внутри круга.
\paragraph*{ДЗ 1.2.8} Задан круг радиуса $R$ с центром в точке $(x_r; y_r)$ и $n$ точек $(x_1; y_1), ... (x_n; y_n)$. Проверить, что все точки находятся вне круга.
\paragraph*{ДЗ 1.2.9} Даны натуральные числа $n, p$ и целые числа $a_1, a_2, ..., a_n$. Вывести количество отрицательных чисел в последовательности $a_1, ..., a_p$ и сумму положительных чисел всей последовательности.
\paragraph*{ДЗ 1.2.10} Даны натуральное число $n$ и целые числа $a_1, a_2, ..., a_n$. Для последовательности $a_1, a_2, ..., a_n$ вывести значения минимума $min(a_1, a_2, ..., a_n)$, максимума $max(a_1, a_2, ..., a_n)$, сумму всех членов $s = \sum \limits_{i=1}^{n}{a_i}$ и среднее арифметическое $\frac {s} {n}$.


\clearpage
\subsubsection*{ЛР №1.2}
\paragraph*{ЛР 1.2.1.1} Дано натуральное число $n$. Вывести все числа $p_i < n$, которым кратно $n$.
\paragraph*{ЛР 1.2.1.2} Дано натуральное число $n$. Является ли $n$ простым числом?
\paragraph*{ЛР 1.2.1.3} Протабулировать функции $x^2$, $\sin x$ и $x \sin x$ для $x$, изменяющегося в диапазоне $[a; b]$ с шагом $\Delta x$.
\paragraph*{ЛР 1.2.1.4} Протабулировать функции $\sqrt{x}$, $\cos x$ и $\frac {sin x} {x} $ для $x$, изменяющегося в диапазоне $[a; b]$ с шагом $\Delta x$.
\paragraph*{ЛР 1.2.1.5} Банкомат содержит купюры номиналом 10, 20, 50 и 100. Вывести, сколько купюр каждого номинала следует выдать для запрашиваемой суммы $n$ ($n$ кратно 10). Число купюр должно быть оптимальным, например для 260 ответ может быть: $100 - 2, 50 - 1, 10 - 1$.
\paragraph*{ЛР 1.2.2.1} Дано натуральное число $n$ и последовательность символов $c_1, c_2, ..., c_n$. Вычислить, сколько раз в последовательности встречается символ $x$.
\paragraph*{ЛР 1.2.2.2} Дано натуральное число $n$ и последовательность строчных букв английского алфавита $c_1, c_2, ..., c_n$ (символы из множества a..z). Каких букв в последовательности больше: гласных или согласных?
\paragraph*{ЛР 1.2.2.3} Даны три строчные буквы английского алфавита $x_1, x_2, x_3 $. Сдвинуть каждую букву на $n \ge 0$ позиций вперёд по алфавиту (циклически). Например, сдвиг на 2 позиции $(a, c, z) \xrightarrow{n=2} (c, e, b)$.
\paragraph*{ЛР 1.2.2.4} Даны три строчные буквы английского алфавита $y_1, y_2, y_3 $. Сдвинуть каждую букву на $n \ge 0$ позиций назад по алфавиту (циклически). Например, сдвиг на 2 позиции $(c, e, b) \xrightarrow{n=2} (a, c, z)$.
\paragraph*{ЛР 1.2.2.5} Дано натуральное число $n$ и последовательность символов $c_1, c_2, ..., c_n$. Найти максимальное количество подряд идущих одинаковых символов.



%\subsubsection*{РК №1}
% TODO: одинаковый размер
\clearpage
\subsection*{Рубежный контроль №1}
\subsubsection*{Вариант №1}
\paragraph*{Задача №1.} Даны три числа $x, y, z$. Найти $|min(x, y) - max(y, z)|$.
\paragraph*{Задача №2.} Даны натуральное число $n$ и последовательность целых чисел $a_1, a_2, ..., a_n$. Вычислить произведение всех чётных членов последовательности.

\subsection*{Рубежный контроль №1}
\subsubsection*{Вариант №2}
\paragraph*{Задача №1.} Даны действительные числа $x, y$. Если $x > y$, то вычислить $\frac { x (y - 2) } { \sqrt { x - y } } $; иначе найти $max(x, 2y)$.
\paragraph*{Задача №2.} Найти максимальное $f_{max}$ значение функции $f(x)=x \sin x$ на интервале $[a; b]$ с точностью вычислений $e>0$.

\subsection*{Рубежный контроль №1}
\subsubsection*{Вариант №3}
\paragraph*{Задача №1.} Дано действительное число $x$ и функция $f(x) = \frac { \sqrt { (x - 3)(x - 1) } } {x - 2}$. Если $x$ входит в область определения функции $f$, что вычислить значение $f(x)$; иначе вывести соответствующее сообщение.
\paragraph*{Задача №2.} Даны натуральные числа $n, p$ и целые числа $b_1, b_2, ..., b_n$. Вычислить сумму всех членов последовательности, кратных $p$.

\subsection*{Рубежный контроль №1}
\subsubsection*{Вариант №4}
\paragraph*{Задача №1.} Дано действительное число $x_0$ и функция $f(x) = \frac { \log_{10} x } {(x+1)(x+2)}$. Если $x_0$ входит в область определения функции $f$, что вычислить значение $f(x_0)$; иначе вывести соответствующее сообщение.
\paragraph*{Задача №2.} Найти минимальное $f_{min}$ значение функции $f(x)=\frac{1}{2} \cos^2 x$ на интервале $[a; b]$ с точностью вычислений $e>0$.

\clearpage
\subsection*{Рубежный контроль №1}
\subsubsection*{Вариант №5}
\paragraph*{Задача №1.} Задан круг радиуса $r$, квадрат с длиной стороны $a$ и прямоугольный треугольник, длины катетов которого равны $b_1, b_2$. Вывести название фигуры с наибольшей площадью. (Если таких фигур несколько, выбрать любую.)
\paragraph*{Задача №2.} Вывести таблицу значений функции $f(x) = \sin^2 x \cos x$ на интервале $[a; b]$ с шагом вычислений $\Delta x$.

\subsection*{Рубежный контроль №1}
\subsubsection*{Вариант №5}
\paragraph*{Задача №1.} Задан круг радиуса $r$, квадрат с длиной стороны $a$ и прямоугольный треугольник, длины катетов которого равны $b_1, b_2$. Вывести название фигуры с наибольшей площадью. (Если таких фигур несколько, выбрать любую.)
\paragraph*{Задача №2.} Вывести таблицу значений функции $f(x) = \sin^2 x \cos x$ на интервале $[a; b]$ с шагом вычислений $\Delta x$.

\subsection*{Рубежный контроль №1}
\subsubsection*{Вариант №5}
\paragraph*{Задача №1.} Задан круг радиуса $r$, квадрат с длиной стороны $a$ и прямоугольный треугольник, длины катетов которого равны $b_1, b_2$. Вывести название фигуры с наибольшей площадью. (Если таких фигур несколько, выбрать любую.)
\paragraph*{Задача №2.} Вывести таблицу значений функции $f(x) = \sin^2 x \cos x$ на интервале $[a; b]$ с шагом вычислений $\Delta x$.

\subsection*{Рубежный контроль №1}
\subsubsection*{Вариант №5}
\paragraph*{Задача №1.} Задан круг радиуса $r$, квадрат с длиной стороны $a$ и прямоугольный треугольник, длины катетов которого равны $b_1, b_2$. Вывести название фигуры с наибольшей площадью. (Если таких фигур несколько, выбрать любую.)
\paragraph*{Задача №2.} Вывести таблицу значений функции $f(x) = \sin^2 x \cos x$ на интервале $[a; b]$ с шагом вычислений $\Delta x$.


\end{document}