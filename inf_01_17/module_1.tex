\documentclass[12pt,a4paper]{report}

\usepackage[left=2cm,right=2cm,top=3cm,bottom=3cm,bindingoffset=0cm,pdftex]{geometry}
\usepackage[utf8]{inputenc}
\usepackage[russian]{babel}
\usepackage{indentfirst}

\usepackage{color}
\usepackage{listings}
\usepackage{amsmath}
%\usepackage{textcomp} % Для значка градуса
\usepackage{amssymb}  % Для значка треугольника
\usepackage{amsmath}

\begin{document}
\parindent=1cm
\pagestyle{empty}

\lstset{ language=Pascal, basicstyle=\small\ttfamily, numbers=left, numberstyle=\tiny, stepnumber=1, numbersep=5pt, extendedchars=\true, showstringspaces=false, breakatwhitespace=true, frame=single, keepspaces=true }
\clearpage

Знакомство с системой программирования Delphi (консольный режим)
Программа (программный код) представляет собой последовательность символов, к числу которых относятся буквы, цифры, знаки препинания, знаки операций.


\subsubsection*{Структура программы}
Простейшая программа
\begin{verbatim}
program HelloWorld;

begin
  WriteLn('Hello, world!');
  ReadLn;
end.
\end{verbatim}


\subsubsection*{Основные стандартные типы данных}
Integer
Real
Char
String

\subsubsection*{Оператор присваивания}
=

\subsubsection*{Выражения}

\subsubsection*{Процедуры ввода-вывода}



\clearpage
\subsubsection*{Разработка программ линейной структуры.}
Программа вычисления площади круга по заданному значению радиуса. 
\begin{verbatim}
program CircleSquare;

{
    Вычисление площади круга по заданному радиусу R
}

const
  MY_PI = 3.14; // Константа MY_PI, её значение не может быть изменено

var
  R: Real; // Переменная для значения радиуса
  S: Real; // Площадь круга

begin
    Write('Input radius: '); // Вывод подсказки для пользователя (что вводим)
    ReadLn(R);               // Ввод с клавиатуры значения радиуса 

    S := MY_PI * R * R;      // Вычисление площади круга по известной формуле

    WriteLn('Circle square is ', S:0:2); // Вывод результата

    ReadLn; // Ожидание нажатия любой клавиши,
            // чтоб программа не завершилась раньше времени
            // и показала результат
end.
\end{verbatim}


\clearpage
Программа для вычисления квадратного уравнения
\begin{verbatim}
program SquareEquation;

var
  A, B, C, D, X1, X2: Real;
begin
  WriteLn('Введите значения коэффициентов');
  Write('A: '); ReadLn(A);
  
  if A = 0 then
  begin
    // Код внутри данного блока begin-end
    // будет выполнен только если A = 0
    WriteLn('A = 0');
    ReadLn;
    Exit; // Команда выхода из программы
  end;
  
  Write('B: '); ReadLn(B);
  Write('C: '); ReadLn(C);

  D := Sqr(B) - 4 * A * C; // Вычисление дискриминанта

  if D < 0 then
  begin
    // Этот код выполнится только если дискриминант будет меньше нуля
    WriteLn('Корни мнимые');
  end
  else
  begin
    // Этот блок выполнится если дискриминант не меньше 0 (больше, либо равен)
    if D = 0 then
    begin
      X1 := -B / (2 * A);
      WriteLn('Один корень, X = ', X1:0:2);
    end
    else
    begin
      X1 := (-B + Sqrt(D)) / 2 / A;
      X2 := (-B - Sqrt(D)) / 2 / A;
      WriteLn('X1 = ', X1:0:2, ', X2 = ', X2:0:2);
    end;
  end;

  ReadLn;
end.
\end{verbatim}




\clearpage
\subsubsection*{ДЗ №1.1}
\paragraph*{ДЗ 1.1.1} Треугольник $\Delta ABC$ на плоскости задан координатами своих вершин. Найти периметр и площадь этого треугольника. Определить скалярное произведение векторов $\overrightarrow{AB}$ и $\overrightarrow{BC}$.
\paragraph*{ДЗ 1.1.2} Даны целые числа $h,m$ ($0 < h \le 12, 0 < m \le 60$), указывающие момент времени "$h$ часов $m$ минут". Определить наименьшее время (число полных минут) до момента, когда часовая и минутная стрелки на циферблате совпадут.
\paragraph*{ДЗ 1.1.3} Даны три точки $A, B, C$ на плоскости, заданные координатами своих вершин. Определить, расположены ли они на одной прямой. Если нет, то вычислить угол $ABC$.
\paragraph*{ДЗ 1.1.4} Даны два числа $a, b$. Если они одновременно чётные или одновременно нечётные, то вывести абсолютное значение (модуль) разности этих чисел; иначе, вывести квадрат их произведения. Вывести наибольшее из чисел $a, b$.
\paragraph*{ДЗ 1.1.5} Даны два числа $x_1, x_2$ и функция $f(x)$, заданная в виде:
\begin{equation*}
 f(x) = 
 \begin{cases}
   e^x, x \le 0; \\
   tg(x), 0 < x \le 1; \\
   lg(x), 1 < x.
 \end{cases}
\end{equation*}
Из $x_1, x_2$ выбрать то значение, при котором значение функции наибольшее.
\paragraph*{1.1.6} Даны действительные числа $x, y$. Если $x$ и $y$ отрицательные, то каждое значение заменить его модулем. Если отрицательно только одно из них, то оба значения увеличить на 0.5. Во всех остальных случаях уменьшить оба значения в 10 раз.
\paragraph*{1.1.7} Даны действительные числа $x, y, z, x > 0, y > 0, z > 0$. Выяснить, существует ли треугольник с такими сторонами. Если существует, то вывести величину наибольшего угла (в градусах).
\paragraph*{1.1.8} Даны действительные числа $x_1, y_1, x_2, y_2, x_3, y_3$. Принадлежит ли начало координат треугольнику с вершинами ($x_1;y_1$), ($x_2;y_2$), ($x_3;y_3$)?
\paragraph*{1.1.9} Даны действительные числа $x, y$ ($x \neq y$). Меньшее из этих двух чисел заменить их полусуммой, а большее -- их удвоенным произведением. 
\paragraph*{1.1.10} Дана комната, представляющая собой параллелепипед с длиной $a$, шириной $b$ и высотой $h$, заданных в метрах (наличием окон, дверей и других проёмов пренебречь, считая плоскость стен непрерывной). Сколько нужно рулонов обоев для того, чтобы оклеить стены этой комнаты целиком? Каждый рулон имеет длину 10 метров и ширину $s$ метров.


\clearpage
\subsubsection*{ЛР №1.1}
\paragraph*{ЛР 1.1.1} Даны действительные числа $x, y, z$. Вычислить $max(x+y+z, xyz)$.
\paragraph*{ЛР 1.1.2} Даны действительные числа $x, y$. Вычислить z:
\begin{equation*}
 z = 
 \begin{cases}
   x - y, \text{ при } x > y, \\
   y - x + 1, \text{ в противном случае}.
 \end{cases}
\end{equation*}
\paragraph*{ЛР 1.1.3} Даны действительные числа $x, y, z$. Выбрать из них те, которые принадлежат интервалу $[a; b)$.
\paragraph*{ЛР 1.1.4} Даны действительные числа $x, y, z$. Выбрать из них те, которые не входят в интервал $[c; d)$.
\paragraph*{ЛР 1.1.5} 

\paragraph*{ЛР 1.2.1} Вычислить площадь кольца с внутренним радиусом $r_1$ и внешним $r_2$.
\paragraph*{ЛР 1.2.2} Указаны длина гипотенузы $x$ и длина одного из катетов $y$ прямоугольного треугольника. Вычислить площадь этого треугольника.
\paragraph*{ЛР 1.2.3} Найти периметр треугольника, заданного координатами своих вершин.
\paragraph*{ЛР 1.2.4} 
\paragraph*{ЛР 1.2.5} 


\clearpage
\subsubsection*{ДЗ №1.2}
\paragraph*{1.2.1}
\paragraph*{1.2.2}
\paragraph*{1.2.3}
\paragraph*{1.2.4}
\paragraph*{1.2.5}
\paragraph*{1.2.6}
\paragraph*{1.2.7}
\paragraph*{1.2.8}
\paragraph*{1.2.9}
\paragraph*{1.2.10}



\end{document}